\documentclass[
  a4paper,
  12pt,
  oneside,
  parskip=half,
  headsepline,
]{scrartcl}

\usepackage{setspace}
\onehalfspacing
\usepackage[utf8]{inputenc}
\usepackage[ngerman]{babel}
\usepackage{natbib}
\bibliographystyle{dinat}
\pagestyle{headings}                % erzeugt eine Kopfzeile

\usepackage{amssymb}
\usepackage{amsmath}
\usepackage[autostyle=true,german=quotes]{csquotes}
\usepackage{multirow}
\usepackage{rotating}
\usepackage[skip=2pt,font=footnotesize]{caption}
\usepackage{booktabs}
\usepackage{xcolor}
\usepackage{url}
\usepackage{subfigure}
\usepackage{textcomp}
\usepackage{pgfplots}
\pgfplotsset{
	compat=newest,
	width  = \textwidth,
	height = 5cm,
	grid style={dashed,lightgray},
	ymajorgrids = true,
	xmajorgrids = false,
	axis lines*=left,
	legend style={ at={(0.5,-0.15)}, draw=none, anchor=north,legend columns=-1}
}
\usepackage{tikz}
\usetikzlibrary{positioning,shapes,shadows,arrows}
\usetikzlibrary{arrows.meta}
\usetikzlibrary{shadows.blur}
\usetikzlibrary{decorations.pathmorphing}
\usepackage{listings}
%\usepackage{geometry}
% \geometry{
% a4paper,
% left=20mm,
% right=20mm,
% top=20mm
% }


\definecolor{c1}{HTML}{80C0ff}
\definecolor{c2}{HTML}{80ffC0}
\definecolor{c3}{HTML}{C0ff80}
\definecolor{c4}{HTML}{ffC080}
%\definecolor{cevp}{HTML}{C080ff}
%\definecolor{c5}{HTML}{80C0ff}
%\definecolor{cevpbt}{HTML}{ff80C0}
%\definecolor{cevp}{HTML}{ff80ff}

\definecolor{c1d}{HTML}{0080ff}
\definecolor{c2d}{HTML}{00ff80}
\definecolor{c3d}{HTML}{80ff00}
\definecolor{c4d}{HTML}{ff8000}
%\definecolor{c2d}{HTML}{8000ff}
%\definecolor{c5d}{HTML}{0080ff}
%\definecolor{c6d}{HTML}{ff0080}
%\definecolor{c7d}{HTML}{ff00ff}

\begin{document}

\title{Master Projekt Bericht\\[1em] \large Erweiterung des Projektes Sempala\\ um das Datenformat Single Table}
\author{Manuel Schneider}
\maketitle
\clearpage
\tableofcontents
\clearpage

\section{Einleitung}

Diese Projektarbeit beschäftigt sich mit der Erweiterung der SPARQL-auf-Hadoop Lö-sung Sempala.
Sempala baut auf SQL auf und verwendet Impala als SQL Backend.
Impala ist eine MPP SQL Query Engine auf Hadoop und erlaubt es SQL Anfragen nahezu in Echtzeit zu bearbeiten.
Sempala übersetzt SPARQL Anfragen in SQL und profitiert von der geringen Latenzzeit der Impala Anfragen.

Im Ausblick seiner Masterthesis \cite{skilevic} schlägt Simon Skilevic ein
Datenmodell vor, das auch, wie im ursprünglichen Sempala Datenformat „Unified
Property Table“ den kompletten Datensatz in einer Tabelle hält. Auch wenn die
Idee als Alternative zum S2RDF Extended Vertical Partitioning, welches auf Spark
basiert, gedacht war, ist das Datenmodell auch auf Impala umsetzbar. In der 
Masterthesis wurde das Konzept Big Table genannt. Um Namenskonflikte zu vermeiden,
wird das Konzept im Rahmen von Sempala \textit{Single Table} genannt.

Wie beim S2RDF ExtVP ist die Idee hinter der Single Table, die Größe der 
Eingabemenge der notwendigen Joins für die Anfragen zu verringern, indem nur 
wirklich notwendige Daten über das Netzwerk übertragen und im Join bearbeitet 
werden müssen.

Im Rahmen dieses Master Projekts soll Sempala um das Datenmodell Single Table 
erweitert werden. Das bedeutet, dass Sempala die Singletable anlegen und anfragen 
können soll. Anschließend soll das neue Datenmodell getestet und evaluiert werden.

In Abschnitt~\ref{sec:model} wird das Datenmodell Single Table im Detail
erklärt. In Abschnitt~\ref{sec:implementation} wird auf die technischen Details
und die praktische Umsetzung eingegangen. In Abschnitt~\ref{sec:eval} werden die
Tests erklärt und die Ergebnisse diskutiert. Abschließend werden
in Abschnitt~\ref{sec:conclusion} die Erkenntnisse zusammengefasst.

\clearpage
\section{Das Datenmodell}
\label{sec:model}

Das Ziel der Single Table ist die Verbesserung der Laufzeiten von
Verbundanfragen. Da die Daten dezentral gelagert sind, müssen bei
Verbundanfragen die für den Verbund notwendigen Daten der rechten Tabelle
über das Netzwerk von jedem Knoten an jeden Knoten verteilt werden. Dieser
Vorgang wird Broadcast genannt. Wenn die Menge der zu übertragenden Daten
wächst, kann dieser Vorgang einen nicht unerheblichen Zeitaufwand mit sich
bringen. Daher wird versucht die Daten, die über das Netzwerk gesendet werden,
zu reduzieren. Selektion wird daher durch Impala vor dem Broadcast ausgeführt,
um die Daten auf die gewünschten Attribute zu reduzieren.

Doch nicht allein der Engpass im Netzwerk spielt eine Rolle. Auch die Selektion
des kompletten Datensatzes kann zu Performanceeinbußen führen. Eine erste
Besserung bringt Impalas implizite Partitionierung der Tripel Tabelle nach
Prädikaten. Impala muss dadurch zur Selektion nach Prädikaten nicht die
komplette Tabelle nach Tripeln mit bestimmten Prädikaten durchsuchen, sondern
legt intern für jedes Prädikat eine eigene Datei an, die direkten Zugriff auf
Tripel mit jenem Prädikat zulässt.

Abgesehen von der Selektion der Daten und dem potentiellen Engpass im Netzwerk
stellt die Komplexität des Verbundes eines der größten Laufzeitprobleme dar. Für
den Verbund muss jedes Tripel des linken Verbundpartners mit jedem Tripel des
rechten Verbundpartners verglichen werden. Angenommen die beiden Verbundmengen
sind im Mittel gleich groß, steigt die Komplexität des Verbundes quadratisch mit
der Eingabemenge. Diese Komplexität kann sich rasch zum dominierenden Faktor der
Laufzeit entwickeln, wie später in der Evaluation zu sehen ist.

Das Datenmodell der Sempala Single Table versucht dem entgegenzuwirken, indem
die Eingabemenge weiter reduziert wird. Erreicht wird das dadurch, dass der
Tripel Tabelle zusätzliche Informationen angehängt werden, die durch Selektion
helfen die Menge der Daten, die zum Verbund gebroadcastet werden, zu reduzieren.
Das hilft einerseits das Netzwerk zu entlasten und andererseits die Eingabemenge
zu reduzieren.

Vereinfacht gesagt wird das Tripel als einzelnes betrachtet und beurteilt zu
welcher Relation es zu anderen Prädikaten steht.
Abbildung~\ref{fig:exampleRDFgraph} zeigt einen minimalen Graphen um die Idee 
zu illustrieren. Im RDF Graph sind fünf Tripel zu sehen. Das Tripel (A,~P$_1$,~B)
steht in Verbindung zu den Prädikaten P$_2$, P$_3$, P$_4$ und P$_5$. Umgekehrt
steht zum Beispiel Tripel (D,~P$_3$,~A) nur zu P$_1$ und P$_2$ in Verbindung.

\tikzstyle{bluenode}=[draw,circle,c1,bottom color=c1!50,top color= white, text=black,
                   minimum width=10pt, blur shadow={shadow blur steps=5}]
\begin{figure}[htbp]
	\centering
	\begin{tikzpicture}
	\node (A) at (0,0)   [bluenode] {A};
	\node (B) at (3,0)   [bluenode] {B};
	\node (C) at (-3,-1) [bluenode] {C};
	\node (D) at (-3,1)  [bluenode] {D};
	\node (E) at (6, 1)  [bluenode] {E};
	\node (F) at (6, -1) [bluenode] {F};
	\draw[-latex] (A) -- (B) node [midway, fill=white] {P$_1$};
	\draw[-latex] (A) -- (C) node [midway, fill=white] {P$_2$};
	\draw[latex-] (A) -- (D) node [midway, fill=white] {P$_3$};
	\draw[-latex] (B) -- (E) node [midway, fill=white] {P$_4$};
	\draw[latex-] (B) -- (F) node [midway, fill=white] {P$_5$};
	\end{tikzpicture}
	\caption{Minimaler RDF Beispielgraph.}
	\label{fig:exampleRDFgraph}
\end{figure}

Das verwendete Datenmodell geht noch einen Schritt weiter und bestimmt genau wie
das Tripel zu anderen Prädikaten in Verbindung steht. Dazu wird angegeben,
welche Seiten der anliegenden Tripel, also Subjekt oder Objekt, verbunden sind.
Daraus folgt, dass es vier Gruppen gibt: Tripel die in Subjekt-Subjekt,
Subjekt-Objekt, Objekt-Subjekt und Objekt-Objekt Verbindung stehen. Im Folgenden
werden nur noch die Abkürzungen SS, SO, OS, und OO verwendet. Da OO Beziehungen
in Anfragegraphen nur sehr selten vorkommen, werden sie in der Single Table nicht
verwendet.

Im Beispiel in Abbildung~\ref{fig:exampleRDFgraph} steht das Tripel (A,~P$_1$,~B)
in SS-Verbindung zu P$_2$, in SO-Verbindung zu P$_3$, in OS-Verbindung zu P$_4$ und in 
OO-Verbindung zu P$_5$, welche aber wie erwähnt ignoriert wird. (D,~P$_3$,~A) hingegen
steht in OS-Verbindung zu P$_1$ und in SS-Verbindung zu P$_2$.

Die Relationen werden in der Single Table in Form von Spalten mit booleschen
Werten gespeichert. Für jedes Prädikat werden drei Spalten angelegt. Für jede Form
der Verbindung eine. Das Datenbankschema für den RDF Graphen in
Abbildung~\ref{fig:exampleRDFgraph} würden dann wie folgt aussehen:

$$\scriptstyle{|S|P|O|SS_{P_1}|SO_{P_1}|OS_{P_1}|SS_{P_2}|SO_{P_2}|OS_{P_2}|SS_{P_3}|SO_{P_3}|OS_{P_3}|SS_{P_4}|SO_{P_4}|OS_{P_4}|SS_{P_5}|SO_{P_5}|OS_{P_5}|}$$

Wenn das Tripel in einer bestimmten Form in Verbindung zu einem Prädikat steht,
wird das Feld der entsprechenden Spalte auf \texttt{true} gesetzt, wenn nicht
dann auf \texttt{false}. Die Single Table des minimalen Beispielgraphen sieht
dementsprechend wie in Tabelle~\ref{tab:singletable} dargestellt aus.

Wie in Tabelle~\ref{tab:singletable} zu sehen ist, ist die Matrix dünnbesetzt.
Je geringer die Dichte des Graphen ist, desto dünnbesetzer ist diese Matrix. RDF
Graphen haben üblicherweise einen sehr geringe Dichte. Daraus folgt, dass die
Single Table für gewöhnliche RDF Graphen sehr schwach besetzt ist.
Spaltenorientierte Datenformate wie Parquet\footnote{Apache Parquet.
http://parquet.apache.org/}, die fähig sind sich wiederholende Werte
speichereffizient zu kodieren, können von diesem Format profitieren
und die Single Table mit wenig zusätzlichem Speicherplatz umsetzen.

\setlength{\tabcolsep}{2.5pt}
\begin{table*}[htb!]
	\centering
	\caption{Single Table zum minimalen Beispielgraph in Abbildung~\ref{fig:exampleRDFgraph}.}
	\label{tab:singletable}
	\begin{tabular*}{\textwidth}{ @{\extracolsep{\fill}} ccc ccccc ccccc ccccc }
		\toprule
		\multicolumn{3}{c}{} & \multicolumn{5}{c}{SS} & \multicolumn{5}{c}{SO} & \multicolumn{5}{c}{OS} \\
		\cline{4-8} \cline{9-13} \cline{14-18} \\[-1em]
		S & P & O & P$_1$ & P$_2$ & P$_3$ & P$_4$ & P$_5$ & P$_1$ & P$_2$ & P$_3$ & P$_4$ & P$_5$ & P$_1$ & P$_2$ & P$_3$ & P$_4$ & P$_5$ \\
		\midrule
		A & P$_1$ & B & $\cdot$ & $\checkmark$ & $\cdot$ & $\cdot$ & $\cdot$ & $\cdot$ & $\cdot$ & $\checkmark$ & $\cdot$ & $\cdot$ & $\cdot$ & $\cdot$ & $\cdot$ & $\checkmark$ & $\cdot$ \\
		A & P$_2$ & C & $\checkmark$ & $\cdot$ & $\cdot$ & $\cdot$ & $\cdot$ & $\cdot$ & $\cdot$ & $\checkmark$ & $\cdot$ & $\cdot$ & $\cdot$ & $\cdot$ & $\cdot$ & $\cdot$ & $\cdot$ \\
		D & P$_3$ & A & $\cdot$ & $\cdot$ & $\cdot$ & $\cdot$ & $\cdot$ & $\cdot$ & $\cdot$ & $\cdot$ & $\cdot$ & $\cdot$ & $\checkmark$ & $\checkmark$ & $\cdot$ & $\cdot$ & $\cdot$ \\
		B & P$_4$ & E & $\cdot$ & $\cdot$ & $\cdot$ & $\cdot$ & $\cdot$ & $\checkmark$ & $\cdot$ & $\cdot$ & $\cdot$ & $\checkmark$ & $\cdot$ & $\cdot$ & $\cdot$ & $\cdot$ & $\cdot$ \\
		F & P$_5$ & B & $\cdot$ & $\cdot$ & $\cdot$ & $\cdot$ & $\cdot$ & $\cdot$ & $\cdot$ & $\cdot$ & $\cdot$ & $\cdot$ & $\cdot$ & $\cdot$ & $\cdot$ & $\checkmark$ & $\cdot$ \\
		\bottomrule
	\end{tabular*}
\end{table*}
\setlength{\tabcolsep}{6pt}

\section{Implementierung}
\label{sec:implementation}

Sempala setzt sich aus zwei Komponenten zusammen. Der Sempala Loader ist für das
erstellen der Single Table zuständig. Dafür wird ein Hadoop Cluster wie zum
Beispiel die Cloudera open-source Apache Hadoop Distribution benötigt, auf dem
HDFS und Impala aktiviert ist. Rohdaten, die als Quelle für die zu erstellende
Tabelle dienen, werden auch benötigt und müssen im HDFS gelagert sein. Der
Sempala Translator ist für das Übersetzen von SPAQRL Anfragen in den Imapala SQL
Dialekt zuständig.

Im Rahmen dieser Master Projekt Arbeit wurde der ursprüngliche Sempala Loader,
der die Datenbanken noch mit dem MapReduce Framework erstellt hat, komplett neu
geschrieben. Ziel war es die Datenbanken alleine mit Java und Impala SQL
anstatt mit MapReduce zu erstellen. Zusätzlich wurde der Sempala Loader um das
Datenformat Single Table erweitert.

Der Sempala Translator wurde ebenfalls um das Datenformat Single Table erweitert.
Die ursprüngliche Variante des Semapla Translators diente lediglich dem
Übersetzen der übergebenen Anfragen. Der Sempala Translator wurde zusätzlich um
die Fähigkeit erweitert, die übergebenen und übersetzten Anfragen direkt auf den
Impala Cluster auszuführen.

Des Weiteren wurde das ursprüngliche Build System durch Apache Maven ersetzt,
welches das Dependency Mangement wesentlich vereinfacht. Das Projekt ist nun in
drei Module gegliedert. Sempala Loader und Translator bilden die ersten zwei
Module. Beide sind eigenständige Projekte, die ausführbare Programme erstellen.
Da der Loader und Translator häufig gemeinsam gebraucht werden und die selben
Abhängigkeiten teilen, vereint das dritte Modul letztere in eine ausführbare
Datei. Ob der Loader oder Translator verwendet werden soll, kann in Form von
High Level Commands angegegeben werden.

Um direkten JDBC Anschluss an den Impala Daemon, der auf dem CDH Cluster läuft,
zu bekommen, wird der \textit{Cloudera JDBC Driver for
Impala}\footnote{http://www.cloudera.com/downloads.html} verwendet. Der Cloudera
JDBC Treiber selbst hat viele Abhängigkeiten, die sorgfältig in das Maven
Dependency Management eingepflegt wurden. Lediglich der Impala JDBC Treiber
selbst ist nicht in öffentlichen Repositories erhältlich. Daher muss er mit dem
Maven Install Plugin in ein lokales Repository installiert werden, welches Maven
zum Auflösen der Dependecies verwenden kann. Mehr Informationen dazu befinden 
sich in den einschlägigen Dateinen und Readmes im Quelltext.

\subsection{Sempala Loader}
\label{sec:loader}

Wie erwähnt ist die Aufgabe des Sempala Loaders die Erstellung der RDF
Datenbanken in verschiedenen Formaten. Dem Programm müssen beim Start die
folgenden Informationen übergeben werden: die Adresse des Koordinatorknotens,
der Name der Datenbank, der Name des Datenmodells und der HDFS-Pfad der RDF
Daten. Die Kommandozeilenschnittstelle erlaubt zusätzlich das Programm zu
instruieren eine Präfix Datei zu verwenden um Namensräume in den Rohdaten zu
ersetzen, einen eventuellen Punkt am Ende der Zeilen der Rohdaten zu ignorieren,
Duplikate in der Eingabemenge zu entfernen und das Löschen der temporären
Tabellen auszulassen. Des Weiteren kann man folgende Standardwerte
überschreiben: Standardport '21050' des Impala Dämons, Subjektspaltenname
'subject', Prädikatspaltenname 'predicate', Objektspaltenname 'object',
Feld\-endmarke der Rohdaten '\textbackslash t', Zeilenendmarke der Rohdaten
'\textbackslash n', Name der Ausgabetabelle 'singletable' beziehungsweise
'propertytable' und das Standard Verbundverhalten 'BROADCAST'.

Der Sempala Loader birgt keine komplizierte Architektur. Strukurell ist der
Loader ein eher imperatives Programm das SQL Anfragen an den Impala Cluster sendet.
Die essentielle Arbeit wird per SQL erledigt.

Da Impala SQL in keinem der populären Java SQL Query Builder, wie zum Beispiel
QueryDsl oder jOOQ, vertreten ist, wurde ein unvollständiger Java SQL Query
Builder entwickelt, der alle nötigen SQL Statements abdeckt. Er dient vorrangig
der Reduktion der Fehleranfälligkeit und Lesbarkeit des Quelltextes.

Für die Erstellung der Property Table als auch der Single Table wird eine Triple
Table benötigt. Um diese zu erstellen, wird mit dem HDFS Pfad, der als Parameter
übergeben wurde, eine externe Tabelle erstellt. Externe Tabellen bieten die
Möglichkeit Klartext Daten im HDFS durch Angabe des Formates der Daten als
Tabelle zu verwenden. Mit dieser externen Tabelle wird eine schematisch
identische interne Tabelle erstellt. Dadurch kann die Tabelle nach Prädikaten
partitioniert, mit Parquet formatiert und mit Snappy komprimiert werden. In
diesem Schritt werden auch, falls verlangt, die Namespaces durch Präfixe
ersetzt.

\subsubsection{Property Table}

Die Property Table ist eine subjektorientierte Tabelle. Das Datenbankschema
setzt sich zusammen aus dem Subjekt und den Prädikaten. In den Feldern der
Prädikatspalten befinden sich die Objekte.

Aus der Triple Table wird eine Ausgangstabelle mit einer einzigen Spalte
mit eindeutigen Subjekten erstellt. Ebenso wird eine leere Ergebnistabelle 
erstellt, deren Schema sich aus einer Subjektspalte und je einer Spalte 
für jedes Prädikat zusammensetzt.

Die Daten für die Property Table werden in einer einzigen SELECT Anfrage
erlangt. Für jedes Prädikat wird ein Left Join der Ausgangstabelle mit den
Tripeln, die jenes Prädikat enthalten, über die Subjektspalte ausgeführt. Der
Left Outer Join ist notwendig, da Subjekte nicht verloren gehen sollen, wenn sie
in den Tripeln nicht vorkommen. Die Projektion der Objektspalte ergibt nach
Umbenennung die Propertyspalte des Prädikats, das gejoint wurde. Schließlich wird
diese Select Anfrage im Rahmen einer INSERT-AS-SELECT Statements in die
Ergebnistabelle geschrieben.

\begin{figure*}[htb]
	\centering
	\begin{tabular}{|c|c|c|}
		\hline
		S & P & O \\
		\hline
		S$_1$ & P$_1$ & O$_1$ \\
		S$_1$ & P$_1$ & O$_2$ \\
		S$_1$ & P$_2$ & O$_3$ \\
		S$_1$ & P$_2$ & O$_4$ \\
		\hline
	\end{tabular}
	~
	\begin{tabular}{|c|c|c|}
		\hline
		S & P$_1$ & P$_2$ \\
		\hline
		S$_1$ & O$_1$ & O$_3$ \\
		S$_1$ & O$_2$ & O$_4$ \\
		S$_1$ & O$_1$ & O$_3$ \\
		S$_1$ & O$_2$ & O$_4$ \\
		\hline
	\end{tabular}
	\caption{Beispiel Triple Table und zugehörige Property Table.}
	\label{tab:propertytable}
\end{figure*}

Abbildung~\ref{tab:propertytable} zeigt eine Beispiel Triple Table und die
Property Table, die daraus resultieren würde. In diesem Beispiel wird
ersichtlich, dass die Daten dupliziert werden. O$_1$ und O$_2$ aus den ersten
beiden Tripel der Triple Table werden durch den ersten Left Join in Spalte P$_1$
untergebracht. Beim zweiten Left Join mit den letzten zwei Tripeln werden die
Daten vervielfacht, wie an den sich wiederholenden Objekten in der Property Table
zu sehen ist.

In diesem minimalen Beispiel ist die Problematik des Speicherbedarfs nicht
direkt ersichtlich, aber wenn nur zwei weitere Tripel mit einem dritten Prädikat
hinzugefügt werden, steigt die Zeilenzahl der Property Table auf acht. Somit kann
man die Zeilenzahl exponentiell steigen lassen. Glücklicherweise ist es
in einem RDF Graphen eher die Ausnahme als die Regel, dass ein Subjekt viele
ausgehende Kanten des selben Prädikats hat. Wenn es auch ein künstlicher Worst
Case ist, das theoretische Problem existiert.

\subsubsection{Single Table}

Im Gegensatz zur Property Table ist die Single Table tripelorientiert. Wie die
Ausgabe des Loaders für Single Table auszusehen hat, wurde in
Abschnitt~\ref{sec:model} detailliert erklärt.

Um die Single Table zu erstellen, werden, wie bei der Propterty Table, die
zusätzlichen Informationen mit Left Joins angehängt. Da die Single Table
tripelorientiert ist, dient hier die Triple Table als Ausgangstabelle. Auch hier
wird ein Left Join verwendet, da die Triple Table komplett erhalten bleiben muss, 
auch wenn kein Verbundpartner gefunden wird.

Da für die Erstellung der Spalten durch den Left Join nur die Kanten, also
Prädikate, und die anliegenden Knoten benötigt werden, werden zwei temporäre
Tabellen erstellt, die lediglich alle eindeutigen Objekt-Prädikat
beziehungsweise Subjekt-Prädikat Relationen enthalten. Im Folgenden werden diese
Tabellen OP und SP genannt.

Für eine SS$_{P}$ Verbindung wird der Left Join der Triple Table mit SP über
die Subjektspalten beider Tabellen ausgeführt. Für eine SO$_{P}$ Verbindung wird
der Left Join der Triple Table mit OP über die Subjektspalten der Triple
Table und Objektspalte der OP ausgeführt. Für eine OS$_{P}$ Verbindung wird der
Left Join der Triple Table mit SP über die Objektspalte der Triple Table und
Subjektspalte der SP ausgeführt. Die zusätzliche Joinbedingung, dass das
Prädikat der SP beziehungsweise OP das Prädikat $P$ der gerade erstellten Spalte
sein muss, sorgt dafür, dass das Feld \texttt{NULL} ist, wenn das Tripel der Zeile nicht
in der Relation steht, die die Spalte darstellt.

Was genau in den Feldern steht, wenn sie nicht \texttt{NULL} sind, ist
unerheblich. Relevant ist nur, ob die Tripel in der Relation stehen, für die
jene Spalte steht. Deshalb wird, wenn das Feld \texttt{NULL} ist, zur Selektion
\texttt{false} ausgegeben und sonst \texttt{true}.

Listing~\ref{lst:loader:singletable} zeigt eine mögliche SQL Anfrage, die die
Daten der Single Table berechnet. Sie zeigt, dass die Anfragen abhängig von der
Anzahl der Prädikate sehr umfangreich werden kann. Jedoch sind Prädikate
üblicherweise endlich und relativ klein. Zum Beispiel hat der Datensatz, der
zum Testen der Single Table verwendet wird, 85 Prädikate.

\begin{minipage}{\linewidth}
\begin{lstlisting}[
	language=sql,
	breaklines=true,
	basicstyle=\linespread{0.5}\footnotesize\ttfamily,
	tabsize=2,
	keywordstyle=\color{blue},
	caption={Beispiel Anfrage zur Erstellung der Single Table},
	label=lst:loader:singletable
]
SELECT 
	tt.s, tt.p, tt.o,
	CASE WHEN tss_p1.s IS NULL THEN false ELSE true END AS ss_p1,
	CASE WHEN tso_p1.o IS NULL THEN false ELSE true END AS so_p1,
	CASE WHEN tos_p1.s IS NULL THEN false ELSE true END AS os_p1,
	CASE WHEN tss_p2.s IS NULL THEN false ELSE true END AS ss_p2,
	CASE WHEN tso_p2.o IS NULL THEN false ELSE true END AS so_p2,
	CASE WHEN tos_p2.s IS NULL THEN false ELSE true END AS os_p2,
	CASE WHEN tss_p3.s IS NULL THEN false ELSE true END AS ss_p3,
	CASE WHEN tso_p3.o IS NULL THEN false ELSE true END AS so_p3,
	CASE WHEN tos_p3.s IS NULL THEN false ELSE true END AS os_p3,
	CASE WHEN tss_p4.s IS NULL THEN false ELSE true END AS ss_p4,
	CASE WHEN tso_p4.o IS NULL THEN false ELSE true END AS so_p4,
	CASE WHEN tos_p4.s IS NULL THEN false ELSE true END AS os_p4,
	CASE WHEN tss_p5.s IS NULL THEN false ELSE true END AS ss_p5,
	CASE WHEN tso_p5.o IS NULL THEN false ELSE true END AS so_p5,
	CASE WHEN tos_p5.s IS NULL THEN false ELSE true END AS os_p5,
FROM tripletable tt
	LEFT JOIN sp_relations tss_p1 ON tt.s=tss_p1.s AND tss_p1.p='p1' 
	LEFT JOIN op_relations tso_p1 ON tt.s=tso_p1.o AND tso_p1.p='p1' 
	LEFT JOIN sp_relations tos_p1 ON tt.o=tss_p1.s AND tos_p1.p='p1'
	LEFT JOIN sp_relations tss_p2 ON tt.s=tss_p2.s AND tss_p2.p='p2' 
	LEFT JOIN op_relations tso_p2 ON tt.s=tso_p2.o AND tso_p2.p='p2' 
	LEFT JOIN sp_relations tos_p2 ON tt.o=tss_p2.s AND tos_p2.p='p2'
	LEFT JOIN sp_relations tss_p3 ON tt.s=tss_p3.s AND tss_p3.p='p3' 
	LEFT JOIN op_relations tso_p3 ON tt.s=tso_p3.o AND tso_p3.p='p3' 
	LEFT JOIN sp_relations tos_p3 ON tt.o=tss_p3.s AND tos_p3.p='p3'
	LEFT JOIN sp_relations tss_p4 ON tt.s=tss_p4.s AND tss_p4.p='p4' 
	LEFT JOIN op_relations tso_p4 ON tt.s=tso_p4.o AND tso_p4.p='p4' 
	LEFT JOIN sp_relations tos_p4 ON tt.o=tss_p4.s AND tos_p4.p='p4'
	LEFT JOIN sp_relations tss_p5 ON tt.s=tss_p5.s AND tss_p5.p='p5' 
	LEFT JOIN op_relations tso_p5 ON tt.s=tso_p5.o AND tso_p5.p='p5' 
	LEFT JOIN sp_relations tos_p5 ON tt.o=tss_p5.s AND tos_p5.p='p5'
\end{lstlisting}
\end{minipage}

Da Anfragen diesen Umfangs sehr speicherintensiv sind, wird die Single Table
inkrementell erstellt. Die Ergebnistabelle wird im Voraus erstellt. Die Daten der
Single Table werden dann partitionsweise erstellt und mit einem INSERT-AS-SELECT
Statement in die Ergebnistabelle eingefügt.

\subsection{Sempala Translator}
\label{sec:translator}


Die Aufgabe des Sempala Loaders ist das Übersetzen und das eventuelle Ausführen
einer Menge übergebener SPARQL Anfragen. Dem Programm werden beim Start
mindestens der Name des Datenmodells und der Pfad zu einer Datei oder einem
Order mit Dateien, die SPARQL Anfragen enthalten, übergeben. Zusätzlich kann die
Adresse und Port eines Koordinatorknotens und der Name einer Datenbank angegeben
werden. Sempala versucht dann die angegebenen SPARQL Anfragen auf dieser
Datenbank auszuführen. Des Weiteren kann der Translator, wie der Loader,
instruiert werden die Namensräume durch Präfixe zu ersetzen, die
Ergebnistabellen zu Benchmarkingzwecken direkt nach dem Erstellen zu löschen
oder die SPARQL Algebra Optimierung zu aktivieren.

Der Sempala Translator durchläuft für jede Anfrage, exklusive der Ausführung der
Anfrage, vier Phasen. Die SPARQL Anfrage wird geparst und in eine interne
Objektstruktur überführt. Aus dieser internen Repräsentation wird ein SPARQL
Algebra Baum erstellt, der die Anfrage repräsentiert. Dieser SPARQL Algebra Baum
wird daraufhin in einen Impala SQL Algebra Baum übersetzt und anschließend wird
aus dem Impala SQL Algebra Baum eine SQL Anfrage erstellt, welche nach Bedarf
ausgeführt werden kann.

Phase eins und zwei wird komplett vom Apache Jena Framework übernommen. Jena
gibt einen Algebra Baum zurück, der nach dem Visitor Pattern traversiert werden
kann. Dazu bietet Jena das Interface eines zu besuchenden Objekts \texttt{Op}
und das dazugehörige Interface des Besuchers \texttt{OpVisitor}. Der SPARQL
Algebra Baum besteht aus Klassen, die das Interface \texttt{Op} realisieren.


\tikzstyle{class}=[
rectangle,
draw=black,
text centered,
text=black,
text width=2.5cm,
shading=axis,
bottom color=c2!25,
top color=white,
shading angle=45,
rounded corners=3pt,
blur shadow={shadow blur steps=5}
]
\begin{figure}[htb]
	\centering	
	\begin{tikzpicture}[node distance=0.5cm]
	\normalsize
	
	
	\node (ImpalaOp) [class, bottom color=c3!25,text width=4cm, rectangle split, rectangle split parts=2] {
		\textit{\textless\textless Interface\textgreater\textgreater}\\
		\textbf{\textit{ImpalaOp}}
		\nodepart{second}
		\tiny
		+ visit(ImpalaOpVisitor)
	};
	
	\node[below = of ImpalaOp.south, anchor=north] (ImpalaBase) [class] {\textbf{\textit{ImpalaBase}}};
	
	\node[below left  = 1cm and 0.5cm of ImpalaBase.south, anchor=north east] (ImpalaOp1) [class] {\textbf{\textit{ImpalaOp1}}};
	\node[below right = 1cm and 0.5cm of ImpalaBase.south, anchor=north west] (ImpalaOp2) [class] {\textbf{\textit{ImpalaOp2}}};
	\node[left        = 1cm and 1.0cm  of ImpalaOp1.west, anchor=east] (ImpalaOp0) [class] {\textbf{\textit{ImpalaOp0}}};
	\node[right       = 1cm and 1.0cm  of ImpalaOp2.east, anchor=west] (ImpalaOpN) [class] {\textbf{\textit{ImpalaOpN}}};
	
	\draw[dashed, -angle 60] (ImpalaBase) -- (ImpalaOp);
	\draw[-open triangle 60] (ImpalaOp0.north) -- ++(0,0.5) -| (ImpalaBase.south);
	\draw[-] (ImpalaOp1.north) -- ++(0,0.5);
	\draw[-] (ImpalaOp2.north) -- ++(0,0.5);
	\draw[-open triangle 60] (ImpalaOpN.north) -- ++(0,0.5) -| (ImpalaBase.south);
	
	% Op0
	
	\node[below=of ImpalaOp0]   (ImpalaBGP) [class] {\textbf{\textit{\footnotesize ImpalaBGP}}};
	\node[below=of ImpalaBGP]   (ImpalaBGPPT) [class] {\textbf{\scriptsize ImpalaBGPPT}};
	\node[below=of ImpalaBGPPT] (ImpalaBGPST) [class] {\textbf{\scriptsize ImpalaBGPST}};
	
	\draw[-open triangle 60] (ImpalaBGP) -- (ImpalaOp0);
	\draw[-] (ImpalaBGPPT.east) -- ++(0.3,0);
	\draw[-open triangle 60] (ImpalaBGPST.east) -- ++(0.3,0) |- (ImpalaBGP.east);
	
	% Op1
	
	\node[below=of ImpalaOp1]     (ImpalaDistint) [class] {\textbf{\scriptsize ImpalaDistinct}};
	\node[below=of ImpalaDistint] (ImpalaFilter)  [class] {\textbf{\scriptsize ImpalaFilter}};
	\node[below=of ImpalaFilter]  (ImpalaOrder)   [class] {\textbf{\scriptsize ImpalaOrder}};
	\node[below=of ImpalaOrder]   (ImpalaProject) [class] {\textbf{\scriptsize ImpalaProject}};
	\node[below=of ImpalaProject] (ImpalaReduced) [class] {\textbf{\scriptsize ImpalaReduced}};
	\node[below=of ImpalaReduced] (ImpalaSlice)   [class] {\textbf{\scriptsize ImpalaSlice}};
	
	\draw[-] (ImpalaBGPPT.east) -- ++(0.3,0);
	\draw[-] (ImpalaDistint.east) -- ++(0.3,0);
	\draw[-] (ImpalaFilter.east) -- ++(0.3,0);
	\draw[-] (ImpalaOrder.east) -- ++(0.3,0);
	\draw[-] (ImpalaProject.east) -- ++(0.3,0);
	\draw[-open triangle 60] (ImpalaSlice.east) -- ++(0.3,0) |- (ImpalaOp1.east);
	
	% Op2
	
	\node[below=of ImpalaOp2]     (ImpalaConditional) [class] {\textbf{\tiny ImpalaConditional}};
	\node[below=of ImpalaConditional] (ImpalaJoin)  [class] {\textbf{\scriptsize ImpalaJoin}};
	\node[below=of ImpalaJoin] (ImpalaLeftJoin)  [class] {\textbf{\scriptsize ImpalaLeftJoin}};
	\node[below=of ImpalaLeftJoin] (ImpalaUnion)  [class] {\textbf{\scriptsize ImpalaUnion}};
	
	\draw[-] (ImpalaConditional.west) -- ++(-0.3,0);
	\draw[-] (ImpalaJoin.west) -- ++(-0.3,0);
	\draw[-] (ImpalaLeftJoin.west) -- ++(-0.3,0);
	\draw[-open triangle 60] (ImpalaUnion.west) -- ++(-0.3,0) |- (ImpalaOp2.west);
	
	% OpN
	
	\node[below=of ImpalaOpN] (ImpalaSequence) [class] {\textbf{\scriptsize ImpalaSequence}};
	
	\draw[-open triangle 60] (ImpalaSequence) -- (ImpalaOpN);
	
	\end{tikzpicture}
	\vspace{1em}
	\caption{Die Impala Algebra Klassen.}
	\label{fig:impala_algebra}
\end{figure}

\clearpage

Die Aufgabe des \texttt{AlgebraTransformers}, ist es den SPARQL Algebra Baum in
einen Impala SQL Algebra Baum zu transformieren. Er realisiert das
\texttt{OpVisitor} Interface und kann daher mit Hilfe des
\texttt{AlgebraWalkers} den SPARQL Algebra Baum traversieren. Während der
Traversierung des Baums erstellt der \texttt{AlgebraTransformer} einen
äquivalenten Impala SQL Algebra Baum. Dazu werden eigene Klassen verwendet, die
sich strukturell an den Jena Algebra Klassen orientieren.
Abbildung~\ref{fig:impala_algebra} zeigt die Klassen, aus denen der Impala
Algebra Baum erstellt wird.

In der Abbildung ist zu erkennen, dass ImpalaBase das zu \texttt{Op} äquivalente
\texttt{ImapalaOp} realisiert. Die vier Operationen \texttt{ImpalaOpX}
bilden Abstraktionen über Operationen, die eine bestimmte Anzahl an
Suboperationen halten. Die atomaren Operationen ohne Parameter bilden immer
\texttt{ImpalaBGP}s. Sie sind im Algebra Baum immer die Blätter.

Als Gegenstück zur \texttt{ImpalaOp} gibt es auch ein zum \texttt{OpVisitor}
äquivalentes \texttt{Im\-pa\-la\-Op\-Vi\-si\-tor} Interface. Der
\texttt{ImpalaOpTransformer} realisiert das \texttt{Im\-pa\-la\-Op\-Vi\-si\-tor} Interface
und kann mit dem \texttt{ImpalaOpWalker} den erstelltem Impala Algebra Baum
traversieren und eine entsprechende SQL Anfrage erstellen.

Bei der Übersetzung einer SPARQL Anfrage zur SQL Anfrage unterscheiden sich die
Impala Single Table von der Property Table nur durch die Übersetzung des Basic
Graph Pattern. Daher wurde für die Einführung der Single Table die Klasse
\texttt{ImpalaBGP} zu einer abstrakten Klasse gemacht und zwei neue Subklassen
eingeführt. \texttt{Im\-pa\-la\-BGP\-Pro\-per\-ty\-Ta\-ble} erledigt die Arbeit,
die zuvor das \texttt{ImpalaBGP} erledigt hat, während
\texttt{Im\-pa\-la\-BGP\-Sin\-gle\-Ta\-ble} für die Übersetzung eines Basic
Graph Patterns in einen SQL String, der mit der Single Table kompatibel ist,
zuständig ist.

Welche der beiden Klassen instantiiert und Teil des Impala Algebrabaumes wird,
entscheidet der \texttt{AlgebraTransformer} während der Transformation des
SPARQL Algebra Baumes.

Im Folgenden wird beschrieben, wie die SPAQRL Anfrage in eine SQL Anfrage, die
zum Datenmodell der Single Table passt, übersetzt wird. SPARQL stellt keine
Anforderungen an die Ordnung der Tripel im Basic Graph Pattern. Für die Joins der
Tripel in der Single Table ist die Reihenfolge aber wichtig, denn für einen Join
wird eine Spalte benötigt, die auf Gleichheit geprüft werden kann. Anschaulich 
bedeutet das, dass die Tripel im BGP anliegend sein müssen. Des Weiteren sollen
Partitionen im Graph erkannt werden, die durch das Kreuzprodukt verbunden werden.

Um Partitionen und Reihenfolge zu bestimmen, kommt ein Nachbarschafts Algorithmus
zum Einsatz. Dazu beginnt man bei einem zufälligen Tripel im Graph und findet
anliegende Tripel. Das wird so oft wiederholt, bis es keine anliegende Tripel
mehr gibt. Sind noch Tripel übrig ist der Anfragegraph partitioniert. Die
bisherigen Tripel gehören zur ersten Partition. Man fährt mit den restlichen
Tripeln wie eben beschrieben fort, bis alle Tripel behandelt wurden. Nun sind
alle Partitionen bekannt und die Reihenfolge, in der die Tripel bearbeitet
wurden ist die Join Reihenfolge. Nähere Details können im Quelltext dieses
Projektes eingesehen werden.

Die einzelnen Tripel in einer Partition werden zu eigenständigen Subqueries
verarbeitet. In diesem Schritt werden die zusätzlichen Informationen der
Singletable verwendet. Mit einem zuvor angelegten invertiertem Index, der
Variablen auf Tripel abbildet, werden die Kriterien, die die Daten der
Subquery zu erfüllen haben, ermittelt und zur Selektion der Subquery
hinzugefügt. Hier wird der eingangs im Abschnitt~\ref{sec:model}
beschriebene Performancegewinn erzeugt, denn die strengere Selektion liefert
weniger Daten, die das Netzwerk oder den Joinvorgang belasten.
Abschließend werden die Subqueries über die anliegenden Variablen gejoint 
und die Partitionen einem Cross Join unterzogen.

\section{Evaluation}
\label{sec:eval}

Die Testumgebung in der die Anfragen ausgeführt wurden besteht aus einem Cluster
aus zehn Rechnern. Alle Rechner besitzen einen Intel Xeon E5-2420 Prozessor, der
mit einer Grundfrequenz von 1,9\,GHz taktet und eine maximale
TurboBoost-Frequenz\footnote{Die Intel Turbo-Boost-Technik erhöht dynamisch die
Frequenz eines Prozessors nach Bedarf, indem die Temperatur- und
Leistungsreserven ausgenutzt werden, um bei Bedarf mehr Geschwindigkeit und
andernfalls mehr Energieeffizienz zu bieten. } von 2,4\,GHz hat. Jede Maschine
hat 32\,GB Arbeitsspeicher und zwei 2\,TB Festplatten. Verbunden sind die
Rechner über eine Gigabit Netzwerkverbindung. Impala läuft auf Cluster im Rahmen
der Cloudera open-source Apache Hadoop Distribution CDH Version 5.7.0, die
Hadoop (HDFS) 2.6.0 und  Impala in der Version 2.5.0 liefert.

Die \textit{Waterloo SPARQL Diversity Test Suite} liefert einen Datengenerator
mit dem die Testdaten generiert wurden. Der Generator erlaubt die
ausgegebenen Datenmengen zu skalieren. Die generierten Datensätze enthalten
Vielfache von etwa 105 Tausend N-Tripel. Für das Testen des Datenformates
Single Table wurden Datensätze des Skalierungsfaktors (von nun an \textit{SF})
10, 100, 1000 und 10000 verwendet. Umgerechnet sind das in etwa eine Million bis
zu einer Milliarde Datensätze.

Die Testergebnisse der Sempala Single Table werden mit ähnlichen
SPARQL-auf-Hadoop Systemen verglichen. Zum einen wird die ursprüngliche Version
von Sempala als bisher einziger Impala SPARQL Query Processor als Referenz
hergezogen, zum anderen S2RDF, welches auf Apache Spark basiert und aktuell
eines der schnellsten SPARQL-auf-Hadoop Systeme ist \cite{s2rdf}.

Sempala in der ursprünglichen Version basiert auf dem Datenmodell Property Table.
Die Property Table basiert auf der Idee RDF Daten subjektorientiert zu speichern,
indem das Datenbankschema für das Subjekt und jedes Prädikat eine Spalte enthält.
Die Objekte werden dann in den einzelnen Feldern gespeichert. Das erfordert die
Möglichkeit verschachtelte Daten zu speichern\footnote{Das Datenformat, das
Impala zugrunde liegt, unterstützt zwar verschachtelte Daten, aber Impala
unterstützt letztere erst seit Impala 2.3. Sempala Property Table wurde vor
diesem Release konzipiert und implementiert und verwendet daher einen Workaround
\cite{sempala}.}. Mehr dazu in \cite{sempala}.

S2RDF bietet verschiedene Datenmodelle. Die zuvor erwähnte Performance wird mit
dem Datenmodell ExtVP erreicht, welches auf der Idee basiert für alle
Partitionen beziehungsweise Prädikate die für einen Join notwendigen Tripel im voraus zu
berechnen. Mehr dazu in \cite{s2rdf}. Ext VP wird als Refernz für aktuelle
Systeme dieser Art verwendet. S2RDF Big Table basiert auf dem selben Datenmodell
wie die Single Table und wird als Referenz für dieses Datenmodell auf einer
anderen Engine verwendet. Somit kann beurteilt werden ob Laufzeitunterschiede
vom Datenmodell oder der Engine her rühren.

\begin{table*}[htb]
	\centering
	\caption{Ladezeiten und HDFS-Speicherbedarf im Vergleich mit ähnlichen Systemen.}
	\label{tab:loader:spacetime}
	\begin{tabular*}{\textwidth}{ ll @{\extracolsep{\fill}} rrrr }
		\toprule
		& SF & 10 & 100 & 1000 & 10000 \\
		\midrule
		\multirow{3}{*}{\rotatebox{90}{\scriptsize{Ladezeit}}}
		& Single Table & 564\,s & 855\,s & 5179\,s & 67080\,s \\
		& Property Table & 26\,s & 56\,s & 333\,s & 2782\,s \\
		& ExtVP & 1430\,s & 2418\,s & 9497\,s & 60572\,s \\
		\midrule
		\multirow{6}{*}{\rotatebox{90}{\scriptsize{Speicherbedarf}}}
		& Klartext & 49\,MB & 507\,MB & 5,3\,GB & 54,9\,GB \\
		& Triple Table & 6,9\,MB & 103\,MB & 1,2\,MB & 13,2\,GB \\
		& Single Table & 13,5\,MB & 141\,MB & 1,6\,GB & 22,7\,GB \\
		& Property Table & 13\,MB & 249\,MB & 3,5\,GB & 40,4\,GB \\
		& ExtVP & 231\,MB & 614\,MB & 6,2\,GB & 63,7\,GB \\
		& Big Table & 240\,MB & 419\,MB & 1,6\,GB & 13,6\,GB \\
		\bottomrule
	\end{tabular*}
\end{table*}

Die Ladezeiten und HDFS-Speicherbedarf der Single Table sind in Tabelle
\ref{tab:loader:spacetime} gelistet. Zum Vergleich wurden auch die Ladezeiten
von S2RDF ExtVP und Sempala Property Table, sowie der Speicherbedarf von S2RDF
ExtVP, S2RDF Big Table und Sempala Property Table gelistet.

Die Laufzeiten der Single Table scheinen mit der Eingabemenge ungefähr gleich
schnell zu wachsen wie die der ExtVP. Der Grund ist wie auch bei der ExtVP die
relativ teure Erstellung der Tabelle, bei der viele LeftJoins ausgeführt werden
müssen.

Der Speicherbedarf der Single Table ist relativ gering. Das liegt zum einen
daran, dass die Single Table den Datensatz gemessen an der Zeilenanzahl nicht
vergrößert, sondern lediglich eine konstante Anzahl an Spalten zu der Triple
Table hinzufügt, die boolesche Werte enthalten. Zum anderen ist das
zugrundeliegende Datenformat Parquet\footnote{Apache Parquet.
http://parquet.apache.org/} wie geschaffen für das Datenmodell der Single Table,
da es fähig ist sich wiederholende Werte durch Repetition Levels
speichereffizient zu kodieren \cite{dremel} und Wiederholungen, bedingt durch
die kleine Grundmenge der Booleschen Algebra, die Regel sind. Der Vergleich mit
der Property Table und Big Table zeigt, dass unabhängig der Kodierung und
Komprimierung das Datenmodell eine große Rolle zu spielen scheint.

Im Folgenden werden die Ergebnisse der Sempala Single Table Testläufe
präsentiert. In Abschnitt~\ref{sec:eval:basic} wird die Single Table mit den
Anfragen getestet, die mit der Waterloo SPARQL Diversity Test Suite geliefert
werden. In Abschnitt~\ref{sec:eval:il} wird das Datenmodell mit der
komplementären Testsuite Incremental Linear Testing getestet, welche sich auf 
lineare Anfragegraphen höheren Durchmessers konzentriert.

\subsection{WatDiv Basic}
\label{sec:eval:basic}

\setlength{\tabcolsep}{2.5pt}
\begin{table*}[htb]
	\centering
	\tiny
 	\caption{Laufzeiten und Mittel der WatDiv Basic Anfragen im Vergleich zu ähnlichen Systemen [ms].}
  	\label{tab:runtime:watdiv:basic}
	\begin{tabular*}{\textwidth}{ ll @{\extracolsep{\fill}} rrrrrrrrrrrrrr }
		\toprule
		& Query & L1 & L2 & L3 & L4 & L5 & AM$_L$ & S1 & S2 & S3 & S4 & S5 & S6 & S7 & AM$_S$ \\
		\midrule
		\multirow{4}{*}{\rotatebox{90}{SF10}}
		& Single Table & 591 & 590 & 527 & 534 & 590 & 567 & 1000 & 634 & 618 & 629 & 617 & 582 & 572 & 665 \\
		& Property Table & 786 & 748 & 642 & 642 & 758 & 715 & 1120 & 822 & 764 & 790 & 866 & 714 & 652 & 818 \\
		& Big Table & 262 & 206 & 157 & 158 & 211 & 199 & 844 & 343 & 364 & 297 & 354 & 248 & 277 & 390 \\
		& ExtVP & 164 & 145 & 97 & 95 & 140 & 128 & 478 & 204 & 180 & 190 & 211 & 138 & 141 & 220 \\
		\midrule
		\multirow{4}{*}{\rotatebox{90}{SF100}}
		& Single Table & 640 & 621 & 570 & 575 & 640 & 609 & 1019 & 730 & 661 & 663 & 668 & 617 & 606 & 709 \\
		& Property Table & 750 & 748 & 636 & 646 & 748 & 706 & 1340 & 854 & 756 & 844 & 940 & 764 & 754 & 893 \\
		& Big Table & 265 & 219 & 171 & 166 & 203 & 205 & 893 & 355 & 369 & 331 & 379 & 285 & 289 & 414 \\
		& ExtVP & 168 & 193 & 126 & 107 & 173 & 153 & 562 & 216 & 214 & 221 & 193 & 146 & 164 & 245 \\
		\midrule
		\multirow{4}{*}{\rotatebox{90}{SF1000}}
		& Single Table & 947 & 706 & 866 & 697 & 670 & 777 & 1792 & 910 & 919 & 831 & 858 & 886 & 908 & 1015 \\
		& Property Table & 904 & 746 & 740 & 664 & 752 & 761 & 3130 & 1058 & 862 & 876 & 960 & 848 & 870 & 1229 \\
		& Big Table & 353 & 281 & 292 & 212 & 224 & 272 & 1032 & 454 & 439 & 372 & 429 & 326 & 426 & 497 \\
		& ExtVP & 202 & 196 & 196 & 132 & 162 & 178 & 735 & 294 & 219 & 209 & 199 & 209 & 191 & 294 \\
		\midrule
		\multirow{7}{*}{\rotatebox{90}{SF10000}}
		& Single Table & 4775 & 2372 & 4346 & 2340 & 1110 & 2989 & 7977 & 4218 & 2945 & 2522 & 2968 & 3735 & 4581 & 4135 \\
		& Single Table$_M$ & 4552 & 2347 & 3484 & 2329 & 1068 & 2756 & 6781 & 3996 & 2975 & 2243 & 3004 & 3252 & 3796 & 3721 \\
		& Single Table$_C$ & 4384 & 2208 & 4021 & 2359 & 1050 & 2804 & 7794 & 4156 & 2875 & 2503 & 2858 & 3741 & 3969 & 4019 \\
		& Single Table$_CM$ & 4409 & 2394 & 3425 & 2364 & 1096 & 2738 & 6577 & 3982 & 2972 & 2197 & 2950 & 3167 & 3779 & 3661 \\
		& Property Table & 3938 & 2140 & 3630 & 2616 & 1914 & 2848 & 17386 & 5368 & 2816 & 2442 & 3142 & 2260 & 3476 & 5270 \\
		& Big Table & 728 & 661 & 776 & 502 & 364 & 606 & 2169 & 886 & 734 & 564 & 743 & 530 & 812 & 920 \\
		& ExtVP & 471 & 498 & 549 & 209 & 270 & 399 & 2208 & 607 & 311 & 329 & 260 & 235 & 420 & 624
  	\end{tabular*}

  	\begin{tabular*}{\textwidth}{ ll @{\extracolsep{\fill}} rrrrrrrrrrr }
		\midrule
		& Query & F1 & F2 & F3 & F4 & F5 & AM$_F$ & C1 & C2 & C3 & AM$_C$ & AM$_T$ \\
		\midrule
		\multirow{4}{*}{\rotatebox{90}{SF10}}
		& Single Table & 742 & 861 & 767 & 941 & 767 & 816 & 912 & 1009 & 828 & 916 & 716 \\
		& Property Table & 866 & 1100 & 1180 & 1172 & 988 & 1061 & 1184 & 1354 & 1066 & 1201 & 911 \\
		& Big Table & 642 & 954 & 618 & 985 & 541 & 748 & 1277 & 1332 & 704 & 1104 & 539 \\
		& ExtVP & 370 & 451 & 376 & 461 & 334 & 398 & 535 & 472 & 450 & 486 & 282 \\
		\midrule
		\multirow{4}{*}{\rotatebox{90}{SF100}}
		& Single Table & 784 & 918 & 804 & 990 & 814 & 862 & 1007 & 1002 & 907 & 972 & 762 \\
		& Property Table & 928 & 1174 & 1150 & 1202 & 1072 & 1105 & 1292 & 1656 & 1702 & 1550 & 998 \\
		& Big Table & 696 & 961 & 624 & 1055 & 591 & 785 & 1524 & 1359 & 865 & 1249 & 580 \\
		& ExtVP & 393 & 539 & 385 & 579 & 398 & 459 & 577 & 689 & 688 & 651 & 337 \\
		\midrule
		\multirow{4}{*}{\rotatebox{90}{SF1000}}
		& Single Table & 1101 & 1220 & 1414 & 1575 & 1436 & 1349 & 1689 & 2435 & 3900 & 2675 & 1288 \\
		& Property Table & 1068 & 1704 & 1538 & 1950 & 2545 & 1761 & 2828 & 5992 & 6040 & 4953 & 1804 \\
		& Big Table & 791 & 1114 & 854 & 1202 & 870 & 966 & 1655 & 1620 & 2322 & 1866 & 763 \\
		& ExtVP & 433 & 642 & 638 & 692 & 672 & 615 & 923 & 1460 & 2929 & 1771 & 567 \\
		\midrule
		\multirow{7}{*}{\rotatebox{90}{SF10000}}
		& Single Table & 3569 & 4043 & 8306 & 10184 & 8331 & 6887 & 4247 & 16770 & 7899 & 9639 & 5362 \\
		& Single Table$_M$ & 3207 & 4183 & 7861 & 7307 & 6661 & 5844 & 4223 & 15896 & 5451 & 8523 & 4731 \\
		& Single Table$_C$ & 3614 & 4102 & 7713 & 9692 & 7934 & 6611 & 4753 & 16212 & 8458 & 9808 & 5231 \\
		& Single Table$_CM$ & 3173 & 4101 & 7665 & 7303 & 6593 & 5767 & 4345 & 15567 & 5460 & 8457 & 4676 \\
		& Property Table & 4420 & 9316 & 12090 & 11668 & 19516 & 11402 & 23136 & 39710 & 37462 & 33436 & 10422 \\
		& Big Table & 1206 & 1619 & 2247 & 1936 & 2512 & 1904 & 3224 & 3410 & 12477 & 6370 & 1905 \\
		& ExtVP & 590 & 1226 & 1969 & 1265 & 2254 & 1461 & 2508 & 2740 & 16407 & 7218 & 1766 \\
		 \bottomrule
  	\end{tabular*}
\end{table*}

Die Waterloo SPARQL Diversity Test Suite liefert zwanzig Anfragevorlagen
verschiedener Typen. \textit{Linear Queries} sind gerade Pfade im Graph, während
\textit{Star Queries} Anfragen sind, die der Form eines Sternes gleichen.
\textit{Snowflake Queries} sind verbundene Sternanfragen und \textit{Complex
Queries} eine Kombination aus allen drei Kategorien. Die verwendeten
Anfragevorlagen können in Anhang~\ref{queries:watdivbasic} eingesehen werden.

Die Linear-, Star- und Snowflake Anfragen sind variabel. Die Vorlagen enthalten
eine Variable die durch einen zufällig gewählten Internationalized Resource
Identifiers einer angegebenen Klasse gewählt werden muss. Beispielsweise muss
für die Anfrage L1 (Vgl. Anhang~\ref{queries:watdivbasic:linear}) die Variable
\%v1\% mit einer IRI der Klasse wsdbm:Website ersetzt werden.

Die Anfragen wurden auf den vier Datensätzen mit dem Skalierungsfaktor 10, 100,
1000 und 10000 ausgeführt. Da Sempala ein RDF-auf-SQL System ist, dessen
eigentlicher Zweck die Ausführung von SPARQL Anfragen ist, werden die Ergebnisse
zur Weiterverarbeitung in einer Ergebnistabelle gespeichert.

Um wirklich nur die Laufzeiten zu messen, die das System braucht, um die Daten zu
erlangen, wird dieser Schritt üblicherweise ausgelassen. Deshalb wurde für den
Skalierungsfaktor 10000 eine weitere Testreihe mit einer modifizierten Version
von Sempala ausgeführt, die die Ergebnisse zählt anstatt sie zu speichern. Diese
Modifikation sorgt dafür, dass die Ergebnisse von jedem Host lokal aggregiert
werden und das Ergebnis von nur einem Record an den Koordinatorknoten gesendet
wird. Da die Ergebnisse je nach Anfrage sehr groß werden können, erspart diese
Maßnahme das potentiell sehr lange dauernde Verteilen der Ergebnisse im Cluster.
Zusätzlich entfällt die zeitaufwändige Festplatten Ein- und Ausgabe, die
notwendig für das dezentrale Speichern der Ergebnisse ist. Im Folgenden wird die
modifizierte Version mit einem M im Index referenziert: Single Table$_\text{M}$.

Impala bietet die Möglichkeit durch HDFS Caching Partitionen oder ganze Tabellen
zwischenzuspeichern. Somit kann Impala Daten mit der Geschwindigkeit des
Speicherbusses lesen und schreiben\,\cite{impala}. Zum Vergleich wurden weitere
Testläufe ausgeführt, die auf eine Tabelle im Cache zugreifen. Die Testläufe mit
einer gecachten Tabelle wurden jeweils mit der normalen und der
modifizierten Version von Sempala ausgeführt. Die Ergebnisse aller Testläufe mit
WatDiv Basic Anfragen sind in Tabelle~\ref{tab:runtime:watdiv:basic} gelistet.
Gecachte Testläufe sind in der Tabelle mit einem $C$ im Index gekennzeichnet.

\begin{figure*}[tb]
	\centering
	\begin{tikzpicture}
	\tiny
	\begin{axis}[
	xtick=data,
	xticklabels from table={data/basic_am_10000.txt}{Query},
	enlarge x limits=0.15,
	ylabel={Mittlere Laufzeit [ms]},
	ymode=log,
	ymin=100,
	ymax=100000,
	ybar=3pt, %space
	bar width=6pt
	]
	\addplot[c1d,fill=c1] table [x expr=\coordindex, y={PT}]    {data/basic_am_10000.txt};
	\addplot[c2d,fill=c2] table [x expr=\coordindex, y={STCo}]  {data/basic_am_10000.txt};
	\addplot[c3d,fill=c3] table [x expr=\coordindex, y={STCoCa}]{data/basic_am_10000.txt};
	\addplot[c4d,fill=c4] table [x expr=\coordindex, y={EVPBT}] {data/basic_am_10000.txt};
	\legend{Property Table, Single Table, Single Table$_C$, Big Table}
	
	\end{axis}
	\end{tikzpicture}
	\caption{Mittlere Laufzeiten der WatDiv Basic Anfragen.}
	\label{fig:runtime:watdiv:basic:am}
\end{figure*}

Die mittleren Laufzeiten der jeweiligen Anfragenkategorie sind in
Abbildung~\ref{fig:runtime:watdiv:basic:am} gegenüber gestellt. Da sich die
Laufzeiten teils um einige Größenordungen unterscheiden, wurde eine
logarithmische Skala gewählt. Bei Betrachtung fällt direkt die Dominanz des
Extended Vertical Partitioning und der zu der Single Table äquivalenten
Implementierung S2RDF Big Table ins Auge. Diese Dominanz ist auch konsistent in
allen Kategorien zu beobachten. Die Ordnung der Ergebnisse legt die Vermutung
nahe, dass Spark die schnellere MPP Engine ist. Diese Vermutung deckt sich auch
mit den Big Data Benchmarks\footnote{https://amplab.cs.berkeley.edu/benchmark/}
der Berkeley Universität von Californien. Um Impala und Spark im Rahmen dieser
Evaluation zu vergleichen, müssen Konfigurationen gewählt werden, die
vergleichbar sind. Um Spark mit der Sempala Single Table zu vergleichen, muss
S2RDF Big Table verwendet werden, da das S2RDF ExtVP ein anderes Datenmodell
verwendet. Da Spark ein In-Memory Sytem ist und zur Evaluation der S2RDF
Big Table die Ergebnisse nach der Anfrage verworfen wurden, muss mit der
gecachten und modifizierten Single Table Count Variante verglichen werden. Die
Laufzeiten zeigen: Spark kann die WatDiv Basic Anfragen etwa zwei bis fünf mal
schneller verarbeiten.

Beim Vergleich der beiden Impala Systeme Sempala Property Table und Sempala
Single Table stellt sich heraus, dass die Single Table im Mittel die schnellere
Alternative ist. Auffällig ist, dass die relative Performance der
Property Table bei den Linear Queries verglichen mit den anderen Kategorien etwas
höher, während die der Single Table etwas niedriger ist. Erstaunlich ist dabei,
dass dieses Verhalten eigentlich bei den Star Queries zu erwarten wäre, weil das
Datenformat Property Table bei einzelnen Stern-Anfragen keine Joins
benötigt\footnote{Das gilt aktuell allerdings nur solange Prädikate in den BGPs
einmalig vorkommen\,\cite{sempala}, was in den WatDiv Basic Star Queries der
Fall ist (Vgl. Anhang \ref{queries:watdivbasic}).}. Des Weiteren wäre auch zu
erwarten gewesen, dass Sempala Single Table in den Linear Queries relativ gut
abschneidet, weil die Architektur der Single Table mit dem Gedanken Linear
Queries zu optimieren entwickelt wurde.

Ein möglicher Faktor für diese Erwartungsuntreue ist die Gestalt der Linear
Queries. Die subjektorientierte Property Table profitiert von den Linear Queries
die aus einer degenerieten Star Query bestehen oder letztere enthalten. Die
Property Table benötigt maximal einen Join für die Verarbeitung der Linear
Queries, da der Anfragegraph nicht gerichtet ist und somit Subjekte mit
auschließlich ausgehenden Kanten enthält. Für L3 und L4 sind Joins nicht
notwendig, da die Anfrage aus einem Subjekt mit zwei ausgehenden Kanten besteht.
Währendessen muss Sempala Single Table zwei beziehungsweise drei Joins
aufwenden.

\begin{figure*}[tb]
	\centering
	\begin{tikzpicture}
	\tiny
	\begin{axis}[
	xtick=data,
	xticklabels from table={data/basic_raw_10000.txt}{Query},
	enlarge x limits=0.05,
	ylabel={Laufzeit [ms]},
	ymode=log,
	ymin=100,
	ymax=100000,
	ybar=1.5pt, %space
	bar width=2pt,
	]
	\addplot[c1d,fill=c1] table [x expr=\coordindex, y={PT}]    {data/basic_raw_10000.txt};
	\addplot[c2d,fill=c2] table [x expr=\coordindex, y={STCo}]  {data/basic_raw_10000.txt};
	\addplot[c3d,fill=c3] table [x expr=\coordindex, y={STCoCa}]{data/basic_raw_10000.txt};
	\addplot[c4d,fill=c4] table [x expr=\coordindex, y={EVPBT}] {data/basic_raw_10000.txt};
	\legend{Property Table, Single Table, Single Table$_C$, Big Table}
	\end{axis}
	\end{tikzpicture}
	\caption{Laufzeiten der WatDiv Basic Anfragen.}
	\label{fig:runtime:watdiv:basic:raw}
\end{figure*}

In Abbildung~\ref{fig:runtime:watdiv:basic:raw} sind die Laufzeiten der
einzelnen Anfragen gegenübergestellt. Mit einer Außnahme zeigt sich auch hier
die selbe Ordnung der Mittelwerte wie in
Abbildung~\ref{fig:runtime:watdiv:basic:am}. Die Ergebnisse der Anfrage C3
zeigen, dass Sempala Single Table die Anfrage etwa doppelt so schnell wie S2RDF
Big Table verarbeiten kann. Der einzig ersichtliche Grund hierfür ist die starke
Korrelation zur Ergebnismenge. Eindeutigere Anzeichen hierfür wird die folgende
Anfrageklasse, die Incremental Linear Queries, liefern, welche teils erheblich
größere Ergebnismengen ergeben.

Generell kann auch beobachtet werden, dass der Overhead bei kleineren
Skalierungsfaktoren eine große Rolle spielt. So liegen die Laufzeiten der
Anfragen auf dem Datensatz SF10 häufig im Rahmen der Standardabweichnung oder
gar über den mittleren Laufzeiten der Anfragen auf dem Datensatz SF100.
Anschaulich überlappt ein erheblicher Teil der Wahrscheinlichkeitsverteilungen.

\subsection{WatDiv Incremental Linear}
\label{sec:eval:il}

\setlength{\tabcolsep}{2pt}
\begin{table*}[htb]
	\centering
	\tiny
    \caption{Laufzeiten und Mittel der WatDiv IL Anfragen der Pfadlängen 5 bis 10 [ms].}
  	\label{tab:runtime:watdiv:il}
  	\begin{tabular*}{\textwidth}{ ll @{\extracolsep{\fill}} rrrrrrrrrrrrrr }
		\toprule
		&  & IL-1-5 & IL-1-6 & IL-1-7 & IL-1-8 & IL-1-9 & IL-1-10 & AM$_{IL-1}$ & IL-2-5 & IL-2-6 & IL-2-7 & IL-2-8 & IL-2-9 & IL-2-10 & AM$_{IL-2}$ \\
		\midrule
		\multirow{4}{*}{\rotatebox{90}{SF10}}
		& Single Table & 888 & 1074 & 1165 & 994 & 1037 & 1088 & 1041 & 782 & 858 & 900 & 966 & 1027 & 1078 & 935 \\
		& Property Table & 1123 & 1064 & 1174 & 1291 & 1361 & 1444 & 1243 & 1054 & 1050 & 1062 & 1079 & 1081 & 1191 & 1086 \\
		& Big Table & 543 & 628 & 759 & 888 & 995 & 1420 & 872 & 556 & 591 & 702 & 839 & 969 & 1171 & 805 \\
		& ExtVP & 307 & 360 & 390 & 493 & 503 & 701 & 459 & 453 & 334 & 388 & 466 & 495 & 660 & 466 \\
		\midrule
		\multirow{4}{*}{\rotatebox{90}{SF100}}
		& Single Table & 2953 & 4067 & 4709 & 2417 & 2199 & 2312 & 3110 & 1493 & 1723 & 1698 & 1798 & 1845 & 1901 & 1743 \\
		& Property Table & 3643 & 3753 & 3844 & 3919 & 4042 & 4126 & 3888 & 2164 & 2257 & 2290 & 2450 & 2527 & 2644 & 2389 \\
		& Big Table & 845 & 928 & 1126 & 1275 & 1443 & 1804 & 1237 & 1317 & 947 & 1093 & 1198 & 1390 & 1715 & 1276 \\
		& ExtVP & 558 & 604 & 711 & 834 & 875 & 1299 & 814 & 969 & 594 & 654 & 795 & 911 & 1047 & 828 \\
		\midrule
		\multirow{4}{*}{\rotatebox{90}{SF1000}}
		& Single Table & 14118 & 18865 & 22838 & 11773 & 10596 & 10654 & 14807 & 10913 & 8888 & 8755 & 8499 & 8974 & 8849 & 9146 \\
		& Property Table & 29321 & 29684 & 29595 & 29696 & 29658 & 29663 & 29603 & 19357 & 19388 & 19496 & 19867 & 20162 & 20152 & 19737 \\
		& Big Table & 2571 & 2308 & 2640 & 2827 & 3057 & 3293 & 2783 & 6034 & 3041 & 3006 & 3386 & 3845 & 3807 & 3853 \\
		& ExtVP & 1724 & 1745 & 1965 & 2029 & 2185 & 2643 & 2048 & 4944 & 1869 & 1980 & 2114 & 2382 & 2413 & 2617 \\
		\midrule
		\multirow{7}{*}{\rotatebox{90}{SF10000}}
		& Single Table & 235865 & 148567 & 178891 & 148158 & 144950 & 176364 & 172133 & 59607 & 125678 & 84605 & 75862 & 113046 & 111570 & 95061 \\
		& Single Table$_M$ & 212930 & 125441 & 161425 & 154996 & 130530 & 150831 & 156025 & 45675 & 87538 & 73453 & 69693 & 79123 & 82707 & 73031 \\
		& Single Table$_C$ & 172769 & 123045 & 142636 & 115680 & 110874 & 114760 & 129961 & 49887 & 98169 & 65767 & 63897 & 79569 & 74608 & 71983 \\
		& Single Table$_{CM}$ & 193029 & 110812 & 139216 & 121499 & 118094 & 121926 & 134096 & 43566 & 79733 & 71752 & 65806 & 72256 & 75662 & 68129 \\
		& Property Table & 128486 & 131304 & 152730 & 152169 & 153360 & 154272 & 145387 & 61843 & 63501 & 64487 & 76717 & 97933 & 96590 & 76845 \\
		& Big Table & 21273 & 16238 & 19872 & 22364 & 24376 & 22195 & 21053 & 57251 & 25835 & 26848 & 28188 & 30787 & 29562 & 33078 \\
		& ExtVP & 12543 & 12252 & 15062 & 15003 & 15478 & 16124 & 14410 & 41188 & 13276 & 14182 & 15261 & 16313 & 13922 & 19024
  	\end{tabular*}

  	\begin{tabular*}{\textwidth}{ ll @{\extracolsep{\fill}} rrrrrrrrrrrrr }
		\midrule
	  	& & IL-3-5 & IL-3-6 & IL-3-7 & IL-3-8 & IL-3-9 & IL-3-10 & AM$_{IL-3}$ & AM-5 & AM-6 & AM-7 & AM-8 & AM-9 & AM-10  \\
		\midrule
		\multirow{4}{*}{\rotatebox{90}{SF10}}
		& Single Table & 9465 & 12070 & 2696 & 67921 & 4431 & 4861 & 16907 & 3712 & 4667 & 1587 & 23294 & 2165 & 2342 \\
		& Property Table & 2624 & 3155 & 1882 & 12548 & 3454 & 3620 & 4547 & 1600 & 1757 & 1373 & 4973 & 1965 & 2085 \\
		& Big Table & 558 & 1020 & 937 & 7907 & 1390 & 1594 & 2234 & 552 & 746 & 799 & 3211 & 1118 & 1395 \\
		& ExtVP & 382 & 783 & 732 & 10698 & 1118 & 1241 & 2492 & 381 & 492 & 503 & 3885 & 705 & 868 \\
		\midrule
		\multirow{4}{*}{\rotatebox{90}{SF100}}
		& Single Table & 96188 & 125219 & 22903 & 714894 & 37645 & 41267 & 173019 & 33545 & 43669 & 9770 & 239703 & 13896 & 15160 \\
		& Property Table & 19214 & 26118 & 11818 & 111690 & 26654 & 28562 & 37343 & 8340 & 10709 & 5984 & 39353 & 11074 & 15092 \\
		& Big Table & 1215 & 2893 & 1904 & 24739 & 3238 & 3473 & 6244 & 1126 & 1589 & 1374 & 9071 & 2024 & 2330 \\
		& ExtVP & 855 & 2766 & 1978 & 36443 & 3608 & 3244 & 8149 & 794 & 1322 & 1114 & 12691 & 1798 & 1863 \\
		\midrule
		\multirow{4}{*}{\rotatebox{90}{SF1000}}
		& Single Table & 479772 & 620307 & 107386 & 3583029 & 213286 & 230228 & 872335 & 168268 & 216020 & 46326 & 1201100 & 77619 & 83243 \\
		& Property Table & 155298 & 194758 & 93424 & 878232 & 217636 & 231430 & 295130 & 67992 & 81277 & 47505 & 309265 & 89152 & 93748 \\
		& Big Table & 4509 & 12225 & 6855 & 108537 & 10335 & 10360 & 25470 & 4371 & 5858 & 4167 & 38250 & 5746 & 5820 \\
		& ExtVP & 4474 & 12188 & 8552 & 178514 & 13411 & 13405 & 38424 & 3714 & 5267 & 4166 & 60886 & 5993 & 6154 \\
		\midrule
		\multirow{7}{*}{\rotatebox{90}{SF10000}}
		& Single Table & 2167746 & 2854454 & 761754 & 42602898 & 2997969 & 3178982 & 9093967 & 821073 & 1042900 & 341750 & 14275639 & 1085322 & 1155639 \\
		& Single Table$_M$ & 252653 & 144565 & 104940 & 728210 & 202999 & 208068 & 273573 & 170419 & 119181 & 113273 & 317633 & 137551 & 147202 \\
		& Single Table$_C$ & 2141407 & 2427551 & 669212 & 42454373 & 3006077 & 3221576 & 8986699 & 788021 & 882921 & 292538 & 14211317 & 1065507 & 1136981 \\
		& Single Table$_{CM}$ & 224795 & 125154 & 89350 & 733727 & 196108 & 205548 & 262447 & 153797 & 105233 & 100106 & 307011 & 128819 & 134379 \\
		& Property Table & 493016 & 595152 & 365868 & 5649620 & 2026680 & 2462137 & 1932079 & 227782 & 263319 & 194362 & 1959502 & 759324 & 904333 \\
		& Big Table & 33370 & 85228 & 64061 & 1656994 & 130310 & 99129 & 344849 & 37298 & 42434 & 36927 & 569182 & 61824 & 50295 \\
		& ExtVP & 29590 & 87525 & 102971 & 2068100 & 158595 & 141940 & 431454 & 27774 & 37684 & 44072 & 699454 & 63462 & 57329 \\
		\bottomrule
  	\end{tabular*}
\end{table*}
\setlength{\tabcolsep}{6pt}

Die Incremental Linear Testing Anfragen sind ein Menge von Vorlagen die
komplementär zu den WatDiv Basic Anfragen, welche größtenteils Anfragen mit
einem maximalen Durchmesser der Länge drei haben, mit einem Durchmesser
von fünf bis zehn abdecken.

Die IL Anfragen bestehen aus drei Kategorien. Die Anfragen der Kategorie IL-1
und IL-2 sind gebunden, das heißt sie beginnen an einer bestimmten IRI, die, wie
in den Watdiv Basic Anfragen, durch eine Variable  bestimmt wird. Die IL-1
Anfragen beginnen bei einem wsdbm:User, während IL-2  Anfragen bei einem
wsdbm:Retailer beginnen. IL-3 Anfragen sind komplett ungebunden und liefern
daher enorm große Ergebnismengen.

Jede Anfragenkategorie besteht aus sechs Anfragen, welche aus einem linearen
Basic Graph Pattern bestehen. Die Anfragen beginnen mit einem  Basic Graph
Pattern mit einem Durchmesser der Länge fünf. Bis zu einer Länge von zehn wird
für jede weitere Anfrage ein weiteres Triple Pattern an die bestehende
Anfrage angehängt. Die Namen der IL Anfragen bilden sich wie folgt:
IL-\textit{Kategorie}-\textit{Pfadlänge}. Die einzelnen Anfragen können in
Anhang~\ref{queries:watdivil} eingesehen werden.

Die Konfigurationen der Testläufe sind die selben wie bei den WatDiv Basic
Anfragen. Die Incremental Linear Anfragen wurden ebenfalls auf den vier
Datensätzen mit dem Skalierungsfaktor 10, 100, 1000 und 10000 ausgeführt. Ebenso
wurden für den Skalierungsfaktor 10000 wieder die normale und modifizierte
Version mit ein- und ausgeschaltetem HDFS Cache ausgeführt. Die Ergebnisse der
Testläufe befinden sich in Tabelle~\ref{tab:runtime:watdiv:il}.

\begin{figure*}[b]
	\centering
	\begin{tikzpicture}
	\tiny
	\begin{axis}[
	xtick=data,
	xticklabels from table={data/il_am_10000.txt}{Query},
	enlarge x limits=0.1,
	ylabel={Mittlere Laufzeit [ms]},
	ymode=log,
	ymin=10000,
	ymax=10000000,
	ybar=2pt, %space
	bar width=4pt
	]
	\addplot[c1d,fill=c1] table [x expr=\coordindex, y={PT}]    {data/il_am_10000.txt};
	\addplot[c2d,fill=c2] table [x expr=\coordindex, y={STCo}]  {data/il_am_10000.txt};
	\addplot[c3d,fill=c3] table [x expr=\coordindex, y={STCoCa}]{data/il_am_10000.txt};
	\addplot[c4d,fill=c4] table [x expr=\coordindex, y={EVPBT}] {data/il_am_10000.txt};
	\legend{Property Table, Single Table ,Single Table$_C$ , Big Table}
	
	\end{axis}
	\end{tikzpicture}
	\caption{Mittlere Laufzeiten der WatDiv IL Anfragen.}
	\label{fig:runtime:watdiv:il:am}
\end{figure*}

\begin{figure*}[b]
	\centering
	\begin{tikzpicture}
	\tiny
	\begin{axis}[
	xtick=data,
	xticklabels from table={data/il_raw_10000.txt}{Query},
	enlarge x limits=0.05,
	ylabel={Laufzeit [ms]},
	ymode=log,
	ymin=10000,
	ymax=10000000,
	ybar=1.5pt, %space
	bar width=2pt,
	x tick label style={rotate=-45,anchor=north west},
	legend style={ at={(0.5,-0.3)}, draw=none, anchor=north,legend columns=-1}
	]
	\addplot[c1d,fill=c1] table [x expr=\coordindex, y={PT}]    {data/il_raw_10000.txt};
	\addplot[c2d,fill=c2] table [x expr=\coordindex, y={STCo}]  {data/il_raw_10000.txt};
	\addplot[c3d,fill=c3] table [x expr=\coordindex, y={STCoCa}]{data/il_raw_10000.txt};
	\addplot[c4d,fill=c4] table [x expr=\coordindex, y={EVPBT}] {data/il_raw_10000.txt};
	\legend{Property Table, Single Table ,Single Table$_C$ , Big Table}
	\end{axis}
	\end{tikzpicture}
	\caption{Laufzeiten der WatDiv IL Anfragen.}
	\label{fig:runtime:watdiv:il:raw}
\end{figure*}

Die mittleren Laufzeiten der Anfragen der selben Kategorie und Durchmesser sind
in Abbildung~\ref{fig:runtime:watdiv:il:am} dargestellt. Da die Ergebnisse
sich teils um ein bis zwei Größenordnungen unterscheiden, wird auch hier eine
logarithmische Skala verwendet.

Die Ordnung der mittleren Laufzeiten der Kategorie IL-1 gleicht den mittleren
Laufzeiten der Watdiv Basic Linear Anfragen. S2RDF Big Table dominiert und
Sempala Property Table und Single Table sind ungefähr gleich schnell. Ebenso
verhält es sich mit der Kategorie IL-2. Allerdings zeigt die Single Table in
IL-3 eine erstaunliche relative Performance. Auch zeigen die mittleren
Laufzeiten über die Pfadlänge, mit Ausnahme der Pfadlänge acht, kein auffälliges
Verhalten. Einzig das Mittel der Pfadlänge acht AM-8 zeigt die selbe Ordnung wie
AM-IL-3.

Ein Blick in die detailliertere Abbildung~\ref{fig:runtime:watdiv:il:raw} der
mittleren Laufzeiten der einzelnen Anfragen zeigt die außerordentlich schlechte
relative Performance der Big Table bei IL-3-8. Das Problem scheint die 
Menge der Daten zu sein \cite{s2rdf}. IL-3-8 ergibt mit etwa 25 Milliarden
Ergebnissen den maximalen Betrag aller Testläufe. Die Laufzeiten aller
Systeme korrelieren mit dem Betrag der Ergebnismenge, jedoch scheint die
Menge der Daten auf Spark einen größeren Einfluss zu haben als auf Impala.

Interessant ist auch der Vergleich der Systeme innerhalb der Spark Engine. S2RDF
Big Table schneidet im Vergleich zum S2RDF ExtVP besser ab, je größer die
Ergebnismenge ist. Das zeigt, dass für große Ergebnismengen Impala nicht  nur
die besser geeignete Engine ist, sondern auch, dass das tripelorientierte
Datenmodell der Impala Single Table beziehungsweise S2RDF Big Table mit großen
Datenmengen performanter arbeitet als S2RDF ExtVP.

Eine weiter Auffälligkeit ist die Anfrage IL-2-5. Auch hier ist die relative
Performance der Big Table ungewöhnlich schlecht. Bezüglich ExtVP  wird in
\cite{s2rdf} erklärt, dass das darauf zurückzuführen ist, dass ExtVP bei den
aufeinanderfolgenden, identischen Prädikaten der letzten zwei Tripel der Anfrage
IL-2-5 keinen Gewinn durch Selektivität der Joins machen kann. Dieses Problem
hat die Single Table nicht, da auch SO, OS und SS Relationen zu dem Prädikat
selbst gehalten werden und somit auch bei aufeinanderfolgenden, identischen
Prädikaten im BGP eine optimale Selektivität erreicht werden kann. Da dies auch
für die Big Table gilt, scheint der Betrag der Zwischenergebnisse für Spark einen
großen Einfluss zu haben.

Bei grober Betrachtung der Laufzeiten aller IL-3 Anfragen auf SF10000 in
Tabelle~\ref{tab:runtime:watdiv:il} wird der Einfluss der Modifikation der
Sempala Single Table deutlich. Mit bloßem Auge kann man beim Querlesen die
Unterschiede der Größenordnungen erkennen, durch die sich Laufzeiten der
Single Table und die der modifizierten Single Table$_M$ unterscheiden. Hier wird
noch einmal klar, dass der direkte Vergleich der anderen Systeme mit den
unmodifizierten Sempala Single Table CTAS Varianten nicht sinnvoll ist.

Dieser Effekt kann man auch bei der WatDiv Basic C3 Anfrage mit, im Vergleich zu
den IL-3, relativ kleinen Ergebnismengen beobachtet werden. Die ungefähr 42
Millionen Ergebnisse heben die Abweichung der Laufzeit der CTAS Variante im
Vergleich zur COUNT Variante erheblich. Im Mittel sind die Laufzeiten der
Single Table 13\% höher als die der Single Table$_M$. Bei der Anfrage C3 steigt
dieses Verhältnis auf 45\% an.

Über alle Testreihen hinweg war der Einsatz des HDFS Caching kaum zu bemerken.
Das könnte daran liegen, dass die komplette Tabelle in den Kernel Cache passt
und die Daten auch ohne das HDFS Caching durch das Betriebssystem bereit
gehalten werden. Die Vorzüge des HDFS Caching werden spürbarer, wenn die
Datenmenge den freien Cache des Betriebssystems übersteigt und viele Anfragen
parallel laufen. Unter diesen Umständen würde der Cache ständig mit
verschiedenen Daten gefüllt und der Effekt des Caches geht verloren. In diesem
Fall kann mit dem HFDS Cache ein statischer Cache einer häufig gebrauchten
Partiton erzwungen werden.

\section{Fazit}
\label{sec:conclusion}

Die Single Table ist sehr sparsam im Speicherverbrauch. Im Vergleich zur
Property Table wird nur die Hälfte des Speichers verbraucht. Ext VP benötigt
gleich drei mal so viel Speicher. Spielt Speicher eine Rolle, zum Beispiel weil
die komplette Tabelle in den Cache soll, dann ist die Singletable eine gute
Wahl.

Was die Laufzeiten betrifft, ist die Single Table im Vergleich zur Property
Table durchweg eine dominante Alternative. Im Mittel kann die Single Table die
Laufzeit der Property Table schlagen und ist im Bereich sehr großer
Ergebnismengen sogar eine ganze Größenordung schneller. ExtVP und ExtVP BigTable
sind zwar im Mittel schneller, jedoch kann die Single Table im Bereich großer
Ergebnismengen doppelt so schnell antworten wie die ExtVP Big Table und sogar
drei mal so schnell wie das ExtVP.

Diese Situation könnte sich allerdings wieder zugunsten der Property Table
ändern. Seit Impala 2.3 werden Complex Types unterstützt, welche der Property
Table über einige Probleme hinweghelfen können. Mit den verschachtelten Daten
kann erreicht werden, dass der Join die Daten nicht mehr vervielfacht und die
Zeilenanzahl konstant bleibt. Aber auch die Laufzeiten können damit in den Griff
bekommen werden, denn kleinere Daten bedeuteuten kleinere Netzwerkbelastung und
weniger Arbeit für den Join Prozess.

Abschließend kann man sagen, dass das Datenmodell Single Table sich als sehr
guten Ersatz für die Property Table herausstellt. Ebenso kann es unter gewissen
Umständen auch als Alternative für S2RDF in Betracht gezogen werden.

\bibliographystyle{plain}
\bibliography{bibliography}

\clearpage
\twocolumn


\lstset{
	language=sparql,
	breaklines=true,
	basicstyle=\linespread{0.5}\scriptsize\ttfamily,
	commentstyle=\color{gray},
	tabsize=2
}
\appendix
\section{WatDiv Basic Queries}
\label{queries:watdivbasic}

\subsection{WatDiv Linear Queries}
\label{queries:watdivbasic:linear}

\begin{lstlisting}[caption={L1},label=query:L1]
#mapping v1 wsdbm:Website uniform
SELECT ?v0 ?v2 ?v3
WHERE {
 ?v0 wsdbm:subscribes %v1% .
 ?v2 sorg:caption ?v3 .
 ?v0 wsdbm:likes ?v2 .
}
\end{lstlisting}

\begin{lstlisting}[caption={L2},label=query:L2]
#mapping v0 wsdbm:City uniform
SELECT ?v1 ?v2
WHERE {
 %v0% gn:parentCountry ?v1 .
 ?v2 wsdbm:likes wsdbm:Product0 .
 ?v2 sorg:nationality ?v1 .
}
\end{lstlisting}

\begin{lstlisting}[caption={L3},label=query:L3]
#mapping v2 wsdbm:Website uniform
SELECT ?v0 ?v1
WHERE {
 ?v0 wsdbm:likes ?v1 .
 ?v0 wsdbm:subscribes %v2% .
}
\end{lstlisting}

\begin{lstlisting}[caption={L4},label=query:L4]
#mapping v1 wsdbm:Topic uniform
SELECT ?v0 ?v2
WHERE {
 ?v0 og:tag %v1% .
 ?v0 sorg:caption ?v2 .
}
\end{lstlisting}

\begin{lstlisting}[caption={L5},label=query:L5]
#mapping v2 wsdbm:City uniform
SELECT ?v0 ?v1 ?v3
WHERE {
 ?v0 sorg:jobTitle ?v1 .
 %v2% gn:parentCountry ?v3 .
 ?v0 sorg:nationality ?v3 .
}
\end{lstlisting}

\subsection{WatDiv Star Queries}
\label{queries:watdivbasic:star}

\begin{lstlisting}[caption={S1},label=query:S1]
#mapping v2 wsdbm:Retailer uniform
SELECT ?v0 ?v1 ?v3 ?v4 ?v5 ?v6 ?v7 ?v8 ?v9
WHERE {
 ?v0 gr:includes ?v1 .
 %v2% gr:offers ?v0 .
 ?v0 gr:price ?v3 .
 ?v0 gr:serialNumber ?v4 .
 ?v0 gr:validFrom ?v5 .
 ?v0 gr:validThrough ?v6 .
 ?v0 sorg:eligibleQuantity ?v7 .
 ?v0 sorg:eligibleRegion ?v8 .
 ?v0 sorg:priceValidUntil ?v9 .
}
\end{lstlisting}

\begin{lstlisting}[caption={S2},label=query:S2]
#mapping v2 wsdbm:Country uniform
SELECT ?v0 ?v1 ?v3
WHERE {
 ?v0 dc:Location ?v1 .
 ?v0 sorg:nationality %v2% .
 ?v0 wsdbm:gender ?v3 .
 ?v0 rdf:type wsdbm:Role2 .
}
\end{lstlisting}

\begin{lstlisting}[caption={S3},label=query:S3]
#mapping v1 wsdbm:ProductCategory uniform
SELECT ?v0 ?v2 ?v3 ?v4
WHERE {
 ?v0 rdf:type %v1% .
 ?v0 sorg:caption ?v2 .
 ?v0 wsdbm:hasGenre ?v3 .
 ?v0 sorg:publisher ?v4 .
}
\end{lstlisting}

\begin{lstlisting}[caption={S4},label=query:S4]
#mapping v1 wsdbm:AgeGroup uniform
SELECT ?v0 ?v2 ?v3
WHERE {
 ?v0 foaf:age %v1% .
 ?v0 foaf:familyName ?v2 .
 ?v3 mo:artist ?v0 .
 ?v0 sorg:nationality wsdbm:Country1 .
}
\end{lstlisting}

\begin{lstlisting}[caption={S5},label=query:S5]
#mapping v1 wsdbm:ProductCategory uniform
SELECT ?v0 ?v2 ?v3
WHERE {
 ?v0 rdf:type %v1% .
 ?v0 sorg:description ?v2 .
 ?v0 sorg:keywords ?v3 .
 ?v0 sorg:language wsdbm:Language0 .
}
\end{lstlisting}

\begin{lstlisting}[caption={S6},label=query:S6]
#mapping v3 wsdbm:SubGenre uniform
SELECT ?v0 ?v1 ?v2
WHERE {
 ?v0 mo:conductor ?v1 .
 ?v0 rdf:type ?v2 .
 ?v0 wsdbm:hasGenre %v3% .
}
\end{lstlisting}

\begin{lstlisting}[caption={S7},label=query:S7]
#mapping v3 wsdbm:User uniform
SELECT ?v0 ?v1 ?v2
WHERE {
 ?v0 rdf:type ?v1 .
 ?v0 sorg:text ?v2 .
 %v3% wsdbm:likes ?v0 .
}
\end{lstlisting}

\subsection{WatDiv Snowflake Queries}
\label{queries:watdivbasic:flake}

\begin{lstlisting}[caption={F1},label=query:F1]
#mapping v1 wsdbm:Topic uniform
SELECT ?v0 ?v2 ?v3 ?v4 ?v5
WHERE {
 ?v0 og:tag %v1% .
 ?v0 rdf:type ?v2 .
 ?v3 sorg:trailer ?v4 .
 ?v3 sorg:keywords ?v5 .
 ?v3 wsdbm:hasGenre ?v0 .
 ?v3 rdf:type wsdbm:ProductCategory2 .
}
\end{lstlisting}

\begin{lstlisting}[caption={F2},label=query:F2]
#mapping v8 wsdbm:SubGenre uniform
SELECT ?v0 ?v1 ?v2 ?v4 ?v5 ?v6 ?v7
WHERE {
 ?v0 foaf:homepage ?v1 .
 ?v0 og:title ?v2 .
 ?v0 rdf:type ?v3 .
 ?v0 sorg:caption ?v4 .
 ?v0 sorg:description ?v5 .
 ?v1 sorg:url ?v6 .
 ?v1 wsdbm:hits ?v7 .
 ?v0 wsdbm:hasGenre %v8% .
}
\end{lstlisting}

\begin{lstlisting}[caption={F3},label=query:F3]
#mapping v3 wsdbm:SubGenre uniform
SELECT ?v0 ?v1 ?v2 ?v4 ?v5 ?v6
WHERE {
 ?v0 sorg:contentRating ?v1 .
 ?v0 sorg:contentSize ?v2 .
 ?v0 wsdbm:hasGenre %v3% .
 ?v4 wsdbm:makesPurchase ?v5 .
 ?v5 wsdbm:purchaseDate ?v6 .
 ?v5 wsdbm:purchaseFor ?v0 .
}
\end{lstlisting}

\begin{lstlisting}[caption={F4},label=query:F4]
#mapping v3 wsdbm:Topic uniform
SELECT ?v0 ?v1 ?v2 ?v4 ?v5 ?v6 ?v7 ?v8
WHERE {
 ?v0 foaf:homepage ?v1 .
 ?v2 gr:includes ?v0 .
 ?v0 og:tag %v3% .
 ?v0 sorg:description ?v4 .
 ?v0 sorg:contentSize ?v8 .
 ?v1 sorg:url ?v5 .
 ?v1 wsdbm:hits ?v6 .
 ?v1 sorg:language wsdbm:Language0 .
 ?v7 wsdbm:likes ?v0 .
}
\end{lstlisting}

\begin{lstlisting}[caption={F5},label=query:F5]
#mapping v2 wsdbm:Retailer uniform
SELECT ?v0 ?v1 ?v3 ?v4 ?v5 ?v6
WHERE {
 ?v0 gr:includes ?v1 .
 %v2% gr:offers ?v0 .
 ?v0 gr:price ?v3 .
 ?v0 gr:validThrough ?v4 .
 ?v1 og:title ?v5 .
 ?v1 rdf:type ?v6 .
}

\end{lstlisting}

\subsection{WatDiv Complex Queries}
\label{queries:watdivbasic:complex}

\begin{lstlisting}[caption={C1},label=query:C1]
SELECT ?v0 ?v4 ?v6 ?v7
WHERE {
 ?v0 sorg:caption ?v1 .
 ?v0 sorg:text ?v2 .
 ?v0 sorg:contentRating ?v3 .
 ?v0 rev:hasReview ?v4 .
 ?v4 rev:title ?v5 .
 ?v4 rev:reviewer ?v6 .
 ?v7 sorg:actor ?v6 .
 ?v7 sorg:language ?v8 .
}
\end{lstlisting}

\begin{lstlisting}[caption={C2},label=query:C2]
SELECT ?v0 ?v3 ?v4 ?v8
WHERE {
 ?v0 sorg:legalName ?v1 .
 ?v0 gr:offers ?v2 .
 ?v2 sorg:eligibleRegion wsdbm:Country5 .
 ?v2 gr:includes ?v3 .
 ?v4 sorg:jobTitle ?v5 .
 ?v4 foaf:homepage ?v6 .
 ?v4 wsdbm:makesPurchase ?v7 .
 ?v7 wsdbm:purchaseFor ?v3 .
 ?v3 rev:hasReview ?v8 .
 ?v8 rev:totalVotes ?v9 .
}
\end{lstlisting}

\begin{lstlisting}[caption={C3},label=query:C3]
SELECT ?v0
WHERE {
 ?v0 wsdbm:likes ?v1 .
 ?v0 wsdbm:friendOf ?v2 .
 ?v0 dc:Location ?v3 .
 ?v0 foaf:age ?v4 .
 ?v0 wsdbm:gender ?v5 .
 ?v0 foaf:givenName ?v6 .
}
\end{lstlisting}

\newpage
\section{WatDiv Increasing Linear Queries} 

\subsection{WatDiv IL-1 Queries} 

\begin{lstlisting}[caption={IL-1-5},label=query:IL-1-5]
#mapping v0 wsdbm:User uniform
SELECT ?v1 ?v2 ?v3 ?v4 ?v5 WHERE {
 %v0% wsdbm:follows ?v1 .
 ?v1 wsdbm:likes ?v2 .
 ?v2 rev:hasReview ?v3 .
 ?v3 rev:reviewer ?v4 .
 ?v4 wsdbm:friendOf ?v5 .
}
\end{lstlisting}

\begin{lstlisting}[caption={IL-1-7},label=query:IL-1-6]
#mapping v0 wsdbm:User uniform
SELECT ?v1 ?v2 ?v3 ?v4 ?v5 ?v6 WHERE {
 %v0% wsdbm:follows ?v1 .
 ?v1 wsdbm:likes ?v2 .
 ?v2 rev:hasReview ?v3 .
 ?v3 rev:reviewer ?v4 .
 ?v4 wsdbm:friendOf ?v5 .
 ?v5 wsdbm:makesPurchase ?v6 .
}
\end{lstlisting}

\begin{lstlisting}[caption={IL-1-7},label=query:IL-1-7]
#mapping v0 wsdbm:User uniform
SELECT ?v1 ?v2 ?v3 ?v4 ?v5 ?v6 ?v7 WHERE {
 %v0% wsdbm:follows ?v1 .
 ?v1 wsdbm:likes ?v2 .
 ?v2 rev:hasReview ?v3 .
 ?v3 rev:reviewer ?v4 .
 ?v4 wsdbm:friendOf ?v5 .
 ?v5 wsdbm:makesPurchase ?v6 .
 ?v6 wsdbm:purchaseFor ?v7 .
}
\end{lstlisting}

\begin{lstlisting}[caption={IL-1-8},label=query:IL-1-8]
#mapping v0 wsdbm:User uniform
SELECT ?v1 ?v2 ?v3 ?v4 ?v5 ?v6 ?v7 ?v8 WHERE {
 %v0% wsdbm:follows ?v1 .
 ?v1 wsdbm:likes ?v2 .
 ?v2 rev:hasReview ?v3 .
 ?v3 rev:reviewer ?v4 .
 ?v4 wsdbm:friendOf ?v5 .
 ?v5 wsdbm:makesPurchase ?v6 .
 ?v6 wsdbm:purchaseFor ?v7 .
 ?v7 sorg:author ?v8 .
}
\end{lstlisting}

\begin{lstlisting}[caption={IL-1-9},label=query:IL-1-9]
#mapping v0 wsdbm:User uniform
SELECT ?v1 ?v2 ?v3 ?v4 ?v5 ?v6 ?v7 ?v8 ?v9 WHERE {
 %v0% wsdbm:follows ?v1 .
 ?v1 wsdbm:likes ?v2 .
 ?v2 rev:hasReview ?v3 .
 ?v3 rev:reviewer ?v4 .
 ?v4 wsdbm:friendOf ?v5 .
 ?v5 wsdbm:makesPurchase ?v6 .
 ?v6 wsdbm:purchaseFor ?v7 .
 ?v7 sorg:author ?v8 .
 ?v8 dc:Location ?v9 .
}
\end{lstlisting}

\begin{lstlisting}[caption={IL-1-10},label=query:IL-1-10]
#mapping v0 wsdbm:User uniform
SELECT ?v1 ?v2 ?v3 ?v4 ?v5 ?v6 ?v7 ?v8 ?v9 ?v10 WHERE {
 %v0% wsdbm:follows ?v1 .
 ?v1 wsdbm:likes ?v2 .
 ?v2 rev:hasReview ?v3 .
 ?v3 rev:reviewer ?v4 .
 ?v4 wsdbm:friendOf ?v5 .
 ?v5 wsdbm:makesPurchase ?v6 .
 ?v6 wsdbm:purchaseFor ?v7 .
 ?v7 sorg:author ?v8 .
 ?v8 dc:Location ?v9 .
 ?v9 gn:parentCountry ?v10 .
}
\end{lstlisting}

\subsection{WatDiv IL-2 Queries} 

\begin{lstlisting}[caption={IL-2-5},label=query:IL-2-5]
#mapping v0 wsdbm:Retailer uniform
SELECT ?v1 ?v2 ?v3 ?v4 ?v5 WHERE {
 %v0% gr:offers ?v1 .
 ?v1 gr:includes ?v2 .
 ?v2 sorg:director ?v3 .
 ?v3 wsdbm:friendOf ?v4 .
 ?v4 wsdbm:friendOf ?v5 .
}
\end{lstlisting}

\begin{lstlisting}[caption={IL-2-7},label=query:IL-2-6]
#mapping v0 wsdbm:Retailer uniform
SELECT ?v1 ?v2 ?v3 ?v4 ?v5 ?v6 WHERE {
 %v0% gr:offers ?v1 .
 ?v1 gr:includes ?v2 .
 ?v2 sorg:director ?v3 .
 ?v3 wsdbm:friendOf ?v4 .
 ?v4 wsdbm:friendOf ?v5 .
 ?v5 wsdbm:likes ?v6 .
}
\end{lstlisting}

\begin{lstlisting}[caption={IL-2-7},label=query:IL-2-7]
#mapping v0 wsdbm:Retailer uniform
SELECT ?v1 ?v2 ?v3 ?v4 ?v5 ?v6 ?v7 WHERE {
 %v0% gr:offers ?v1 .
 ?v1 gr:includes ?v2 .
 ?v2 sorg:director ?v3 .
 ?v3 wsdbm:friendOf ?v4 .
 ?v4 wsdbm:friendOf ?v5 .
 ?v5 wsdbm:likes ?v6 .
 ?v6 sorg:editor ?v7 .
}
\end{lstlisting}

\begin{lstlisting}[caption={IL-2-8},label=query:IL-2-8]
#mapping v0 wsdbm:Retailer uniform
SELECT ?v1 ?v2 ?v3 ?v4 ?v5 ?v6 ?v7 ?v8 WHERE {
 %v0% gr:offers ?v1 .
 ?v1 gr:includes ?v2 .
 ?v2 sorg:director ?v3 .
 ?v3 wsdbm:friendOf ?v4 .
 ?v4 wsdbm:friendOf ?v5 .
 ?v5 wsdbm:likes ?v6 .
 ?v6 sorg:editor ?v7 .
 ?v7 wsdbm:makesPurchase ?v8 .
}
\end{lstlisting}

\begin{lstlisting}[caption={IL-2-9},label=query:IL-2-9]
#mapping v0 wsdbm:Retailer uniform
SELECT ?v1 ?v2 ?v3 ?v4 ?v5 ?v6 ?v7 ?v8 ?v9 WHERE {
 %v0% gr:offers ?v1 .
 ?v1 gr:includes ?v2 .
 ?v2 sorg:director ?v3 .
 ?v3 wsdbm:friendOf ?v4 .
 ?v4 wsdbm:friendOf ?v5 .
 ?v5 wsdbm:likes ?v6 .
 ?v6 sorg:editor ?v7 .
 ?v7 wsdbm:makesPurchase ?v8 .
 ?v8 wsdbm:purchaseFor ?v9 .
}
\end{lstlisting}

\begin{lstlisting}[caption={IL-2-10},label=query:IL-2-10]
#mapping v0 wsdbm:Retailer uniform
SELECT ?v1 ?v2 ?v3 ?v4 ?v5 ?v6 ?v7 ?v8 ?v9 ?v10 WHERE {
 %v0% gr:offers ?v1 .
 ?v1 gr:includes ?v2 .
 ?v2 sorg:director ?v3 .
 ?v3 wsdbm:friendOf ?v4 .
 ?v4 wsdbm:friendOf ?v5 .
 ?v5 wsdbm:likes ?v6 .
 ?v6 sorg:editor ?v7 .
 ?v7 wsdbm:makesPurchase ?v8 .
 ?v8 wsdbm:purchaseFor ?v9 .
 ?v9 sorg:caption ?v10 .
}
\end{lstlisting}

\subsection{WatDiv IL-3 Queries} 

\begin{lstlisting}[caption={IL-3-5},label=query:IL-3-5]
SELECT ?v0 ?v1 ?v2 ?v3 ?v4 ?v5 WHERE {
 ?v0 gr:offers ?v1 .
 ?v1 gr:includes ?v2 .
 ?v2 rev:hasReview ?v3 .
 ?v3 rev:reviewer ?v4 .
 ?v4 wsdbm:friendOf ?v5 .
}
\end{lstlisting}

\begin{lstlisting}[caption={IL-3-7},label=query:IL-3-6]
SELECT ?v0 ?v1 ?v2 ?v3 ?v4 ?v5 ?v6 WHERE {
 ?v0 gr:offers ?v1 .
 ?v1 gr:includes ?v2 .
 ?v2 rev:hasReview ?v3 .
 ?v3 rev:reviewer ?v4 .
 ?v4 wsdbm:friendOf ?v5 .
 ?v5 wsdbm:likes ?v6 .
}
\end{lstlisting}

\begin{lstlisting}[caption={IL-3-7},label=query:IL-3-7]
SELECT ?v0 ?v1 ?v2 ?v3 ?v4 ?v5 ?v6 ?v7 WHERE {
 ?v0 gr:offers ?v1 .
 ?v1 gr:includes ?v2 .
 ?v2 rev:hasReview ?v3 .
 ?v3 rev:reviewer ?v4 .
 ?v4 wsdbm:friendOf ?v5 .
 ?v5 wsdbm:likes ?v6 .
 ?v6 sorg:author ?v7 .
}
\end{lstlisting}

\begin{lstlisting}[caption={IL-3-8},label=query:IL-3-8]
SELECT ?v0 ?v1 ?v2 ?v3 ?v4 ?v5 ?v6 ?v7 ?v8 WHERE {
 ?v0 gr:offers ?v1 .
 ?v1 gr:includes ?v2 .
 ?v2 rev:hasReview ?v3 .
 ?v3 rev:reviewer ?v4 .
 ?v4 wsdbm:friendOf ?v5 .
 ?v5 wsdbm:likes ?v6 .
 ?v6 sorg:author ?v7 .
 ?v7 wsdbm:follows ?v8 .
}
\end{lstlisting}

\begin{lstlisting}[caption={IL-3-9},label=query:IL-3-9]
SELECT ?v0 ?v1 ?v2 ?v3 ?v4 ?v5 ?v6 ?v7 ?v8 ?v9 WHERE {
 ?v0 gr:offers ?v1 .
 ?v1 gr:includes ?v2 .
 ?v2 rev:hasReview ?v3 .
 ?v3 rev:reviewer ?v4 .
 ?v4 wsdbm:friendOf ?v5 .
 ?v5 wsdbm:likes ?v6 .
 ?v6 sorg:author ?v7 .
 ?v7 wsdbm:follows ?v8 .
 ?v8 foaf:homepage ?v9 .
}
\end{lstlisting}

\begin{lstlisting}[caption={IL-3-10},label=query:IL-3-10]
SELECT ?v0 ?v1 ?v2 ?v3 ?v4 ?v5 ?v6 ?v7 ?v8 ?v9 ?v10 WHERE {
 ?v0 gr:offers ?v1 .
 ?v1 gr:includes ?v2 .
 ?v2 rev:hasReview ?v3 .
 ?v3 rev:reviewer ?v4 .
 ?v4 wsdbm:friendOf ?v5 .
 ?v5 wsdbm:likes ?v6 .
 ?v6 sorg:author ?v7 .
 ?v7 wsdbm:follows ?v8 .
 ?v8 foaf:homepage ?v9 .
 ?v9 sorg:language ?v10 .
}
\end{lstlisting}

\nocite{*}
\end{document}
